\begin{figure}[p!]
    \resizebox{\linewidth}{!}{
        \begin{tikzpicture}
            \node[text width=\linewidth] (T) at (-1, 0.25) {
                    \textsc{\LARGE{Successi di inizio gioco}}\\
                    \textit{Il gioco è vasto e come prima cosa dovresti concentrarti sull'osservare gli altri per imparare a giocare. Dopodiché, la cosa migliore è giocare in maniera ambiziosa e orientata agli obiettivi; cogli l'attimo!}
                };
            \node[draw, ultra thick, rectangle,font=\Large\scshape] (A) at (-1, -1.4) {Spedizione di Successo};
            \node[draw, ultra thick, rectangle, text width=\linewidth] at (-1, -3.3) {\textit{Incassa 100 MO}\\ I personaggi iniziali sono piuttosto poveri per gli standard degli avventurieri. Una singola spedizione di successo può risolvere questo problema, aprendo nuove possibilità: assoldare gregari, acquistare equipaggiamento e dedicare del tempo a preparare la prossima spedizione.};

            \node[draw, ultra thick, rectangle,font=\Large\scshape] (A) at (-1, -5.4) {Raggiungi il 2° Livello};
            \node[draw, ultra thick, rectangle, text width=\linewidth] at (-1, -7.55) {\textit{Porta un personaggio a $\sim$2000 PX.}\\ Come ottenere una cintura nelle arti marziali, è un divisorio formale tra casual e regolari. Nella mia esperienza, ci voglio 10-20 sessioni di tentativi, errori e abilità, salvo gli occasionali colpi di fortuna. Il gioco che pratichi per arrivare al secondo livello è tutto il \dnd{} che conta, il fondamento del gioco di fascia bassa. Tutto il resto si costruisce su queste lezioni.};
            \node[draw, ultra thick, rectangle,font=\Large\scshape] (A) at (-1, -9.9) {Fai tutto};
            \node[draw, ultra thick, rectangle, text width=\linewidth] at (-1, -12.45) {
                \begin{tikzpicture}
                    \node[text width=0.4\linewidth, font=\itshape] at (-1, 0) {Avvia una raccolta di personaggi.\\Pensiona un personaggio.\\Ottieni un seguace.\\Diventa famoso.\\Cavalca un drago.\\Dichiara il tuo Nome.};
                    \node[text width=0.4\linewidth] at (4.5, -0.3) {E, ovviamente, tieni conto dell'obiettivo generale della campagna. Spesso è solamente di portare un avventuriero tanto lontano quanto può arrivare, ma non è il caso di ogni campagna.};
                \end{tikzpicture}
            };
        \end{tikzpicture}
    }
\end{figure}