\begin{figure}[p!]
    \resizebox{\linewidth}{!}{
        \begin{tikzpicture}
            \node[text width=\linewidth, align=center] (T) at (-1, 0.25) {
                    \textsc{\LARGE{Il Muro della Saggezza della Via del Wargaming}}\\
                    \textit{Alcuni detti da attaccare alle pareti del tuo dojo.}
                };
            \node[draw, ultra thick, rectangle,font=\Large\scshape] (A) at (-1, -1.4) {Tre Pietre Angolari};
            \node[draw, ultra thick, rectangle, text width=\linewidth] at (-1, -3.3) {
                \begin{tikzpicture}
                    \node[text width=0.3\linewidth,font=\small] at (-1, 0.25) {
                        \textbf{Arbitro neutrale}\\
                        \textit{
                            Non si bara\\
                            Equilibrio dinamico\\
                            Procedure igieniche\\
                            Partecipazione autentica
                        }
                    };

                    \node[text width=0.3\linewidth,font=\small] at (2.3, 0.25) {
                        \textbf{Gioco simulativo}\\
                        \textit{
                            Ruling, non regole\\
                            Le regole sono residuati\\
                            Insegna la realtà, usa la realtà
                        }
                    };

                    \node[text width=0.3\linewidth,font=\small] at (5.6, 0) {
                        \textbf{Risultati reali}\\
                        \textit{
                            I giocatori vincono e perdono\\
                            I PE sono veri punti\\
                            Imparare la sportività\\
                            Imparare cose
                        }
                    };
                \end{tikzpicture}
            };
            \node[rectangle, text width=\linewidth] at (-1, -5.7) {
                \textit{La definizone dogmatica della via del} wargaming\textit{, in un certo senso. I tre principi chiave sembrano applicarsi all'intera tradizione del} free krieksspiel.
            };

            \node[draw, ultra thick, rectangle,font=\Large\scshape, text width=0.35\linewidth, align=center] at (-4, -7.2) {Gestione del Gruppo};
            \node[draw, ultra thick, rectangle, text width=0.4\linewidth, font=\itshape] at (-4, -10.2) {
                Creazione rapida dei personaggi\\
                Inizio al 1° livello\\
                Nessun background\\
                I personaggi muoiono/si ritirano\\
                Uso di raccolte di personaggi\\
                Uso di mercenari e seguaci
            };

            \node[draw, ultra thick, rectangle,font=\Large\scshape, text width=0.35\linewidth, align=center] at (2, -8) {Modello Simulativo};
            \node[draw, ultra thick, rectangle, text width=0.4\linewidth, font=\itshape] at (2, -11.4) {
                Basi realistiche\\
                Statistiche iniziali casuali\\
                I PE sono basati sugli obiettivi\\
                Il livello modella la statura eroica\\
                Progressione geometrica dei livelli\\
                Crescita lineare dei PF\\
                Crescita sub-lineare dei danni
            };
        \end{tikzpicture}
    }
\end{figure}