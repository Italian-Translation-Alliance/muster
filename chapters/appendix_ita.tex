\chapter{Appendici all'edizione italiana}

\section{A: Avventure citate}\label{appendix:ita:avventure}

Le avventure sono riportate nell'ordine in cui sono menzionate nel testo principale.

\paragraph{\textsc{Keep on the Borderlands}}: Uscita in italiano come "La Rocca sulle Terre di Confine", per Editrice Giochi, nel lontano 1985. Una versione digitale o PoD dell'edizione originale inglese è disponibile in diversi store online, così come la sua reincarnazione per la quarta edizione di \dnd. Non si hanno notizie di una ristampa italiana.

\paragraph{\textsc{The Screams for Jedder's Hole}}: L'avventura, pubblicata sul blog di Dyson Logos, non ha mai avuto un'edizione italiana.

\paragraph{\textsc{Tomb of the Iron God}}: Avventura di Matt Flinch per il retroclone \textsc{Swords\&Wizardry}, mai tradotta in italiano.

\paragraph{\textsc{Temple of the Ghoul}}: Avventura di H. John Martin per il retroclone \textsc{OSRIC}, difficile da trovare online e mai tradotta in italiano.

\paragraph{\textsc{Tower of the Stargazer}}: Avventura di James Edward Raggi IV per il suo retroclone \textsc{Lamentations of the Flame Princess}, mai tradotta in italiano.

\paragraph{\textsc{No Dignity in Death: Three Brides}}: Antologia di tre avventure a tema spose morte, ambientate nella cittadina di Pembrooktonshire per \textsc{Lamentations of the Flame Princess}. Mai tradotta in italiano, l'antologia è ora (Febbraio 2024) disponibile esclusivamente assieme al documento \textsc{People of Pembrooktonshire}.

\paragraph{\textsc{Rahasia}}: Si tratta di un'avventura dalla storia editoriale piuttosto travagliata, scritta da Tracy e Laura Hickman per la loro casa editrice indipendente, Day Star West Media, nel 1980 (vendette circa 200 copie). A seguito del fallimento della Day Star l'avventura venne acquisita dalla TSR, che la stampò inizialmente come "RPGA1 Rahasia", in edizione limitata e riservata ai membri dell'RPGA (Role Playing Game Association, il gruppo di gioco organizzato della TSR, predecessore dell'attuale Adventurers League) e successivamenten come "B7 Rahasia", disponibile al pubblico generalista, quest'ultima versione si può ancora trovare in PDF e PoD sugli store online. Ogni iterazione ha cambiato qualche dettaglio, ma la maggior parte delle recensioni fanno riferimento all'ultima in quanto la più facile da reperire. Non è mai stata tradotta in italiano.

\paragraph{\textsc{Death Frost Doom}}: Una delle più famose avventure per \textsc{Lamentations of the Flame Princess}, scritta originariamente da James Raggi e successivamente aggiornata e riveduta da Zack Sabbath. Non ha mai avuto una traduzione italiana.