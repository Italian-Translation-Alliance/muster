\chapter{Giocare a \dnd come a un Wargame: un manifesto}

\textit{Ritenevo che questo fosse uno di quegli argomenti in cui si parte prima da un'immagine generale. La versione stampata di un'elevator pitch, il manifesto tenta di essere interessante ed informativo pur mantenendo una forma compatta. Se vi piace, è semplice da stampare e piegare su un singolo foglio, da distribuire ai potenziali giocatori.}

\textit{E se proprio volete saperne di più, c'è comunque il resto del libro.}

\clearpage

Quindi ti hanno chiesto di giocare a Dungeons \& Dragons, il primo gioco di ruolo. Ma ci sono innumerevoli varianti e modi di giocarlo là fuori. Esattamente, cosa si fa in questo gioco? Questa è la guida per giocarlo come un wargame.

\usection{La Sfida}

Un piccolo commando entra in un mondo fantastico e sotterraneo.

\textbf{Decifra la logica onirica}: Osserva l'ambiente e applica la tua conoscenza per predire le regole fantastiche della realtà del dungeon.

\textbf{Mappa il labirinto}: Esplora con cura e impara dove si trovano mostri e tesori. Scegli quali porte aprire e che rischi correre.

\textbf{Avanzare o tornare indietro}: Mentre le risorse si riducono e le informazioni si accumululano, dovrai valutare gli obiettivi della spedizione contro il rischio di perdite, scegliendo le manovre con cui avanzare o ritirarti

\usection{L'eredità del Kriegspiel}

\dnd è un gioco di ruolo che è anche un \textit{wargame}. La sua base filosofica rientra nella simulazione dei conflitti della tradizione dei giochi di guerra. I principi su cui insistiamo qui sono una novità solo se comparati con altri tipi di giochi di ruolo.

I giochi di guerra nascono nel diciannovesimo secolo come hobby e strumento di addestramento dei militari. I loro ideali creativi si basano su apprendimento e sportività; giochiamo per capire le dinamiche di un conflitto, imparare cultura e scienza e crescere in questo contesto.

Sebbene oggi questo venga ignorato da molti, \dnd rimane uno dei più alti successi del wargaming. Ha un argomento pressoché unico nel campo, concentrandosi sul mito e sulla leggenda, conflitti con incognite sconosciute e fantastiche e lavoro di squadra su scala da schermaglia.

\usection{Le tre pietre angolari sono il fondamento della via}

\usubsection{Arbitro neutrale}

Il Game Master è un arbitro; il suo ruolo è di preparare uno scenario sfidante e condurlo equamente a qualsiasi risultato arrivi. Insegnare le regole e giudicare gli eventi.

L'arbitro non ha un piano per il risultato della partita. Non è responsabile del risultato e di conseguenza c'è spazio perché i giocatori abbiano agentività.

L'arbitro non è un padrone autoritario, è un funzionario del gioco, non un superiore a livello sociale. State imparando a ballare assieme.

\usubsection{Successi reali}

Questo è un gioco di abilità e coraggio, messi in gioco contro avversità quantificabili. Non è uno spettacolo teatrale; non state giorecitando all'interno della storia pianificata dal Game Master.

Naturalmente, questo significa che anche le perdite sono altrettanto reali. Eccitanti da evitare, deprimenti da fronteggiare. Ma imparerai da esse, proprio perché sono reali.

Successi reali, orgoglio reale, apprendimento reale e sportività reale sono possibili solamente quando il gioco è reale.

\usubsection{Regole di simulazione}

Il gioco è una simulazione di conflitto, il cui scopo è di mostrare come le cose funzionano; insegnare la realtà e usare la realtà. Non è un gioco da tavolo, convenientemente autocontenuto e bilanciato.

I giocatori manovrano contro uno scenario immaginato e il suo funzionamento. L'arbitro esiste per applicare e adattare le regole per riflettere lo scenario.

L'ideale di simulare una realtà, anche una fantastica, è sempre un'aspirazione e un compromesso tra giocabilità e profondità. Il gioco viaggia su un equilibrio d'oro per catturare *qualcosa* di reale e rimanere divertente e fluido. Gli strumenti meccanici adatti allo scopo dipendono dalle nascenti abilità e interessi del gruppo stesso.

\usection{Alcuni principi fondamentali per i giocatori}

\textbf{Fai domande} sulle regole, l'arbitro ha la responsabilità di informarti. \textbf{Fai domande} sulla situazione e il mondo di gioco, l'arbitro ha la responsabilità di informarti. \textbf{Fai domande} agli altri giocatori, dovreste cospirare assieme per sconfiggere lo scenario.

\textbf{Non dare per scontato} che ci sia qualcosa di "giusto" o "corretto" che dovresti fare. Valutare lo scenario è compito tuo e \textbf{la ritirata è sempre un'opzione}.

Per essere sportivo, \textbf{gioca lo scenario} invece di impuntarti sulle regole o di cercare di manipolare il GM. Per essere un grande giocatore \textbf{manovra con coraggio} e prenditene le conseguenze!

\usection{Struttura di una campagna basilare}

Una \textbf{generazione rapida dei personaggi} è fondamentale, perché i personaggi muoiono. Un sacco.

\textbf{Niente background}, conosceremo i personaggi se sopravviveranno.

\textbf{Comincia al livello 1}, sempre, qualsiasi altro livello rovina la prospettiva.

La \textbf{morte dei personaggi} è una necessità, ciò che mantiene il gioco onesto.

Usa un \textbf{roster di personaggi} per ridurre i rischi e scegliere personaggi adeguati ad avventure differenti.

I \textbf{punti esperienza sono basati sugli obiettivi}, garantiti solo in caso di successo. Tesoro, non combattimenti.

I \textbf{punti esperienza sono consistenti} e basati solo sui risultati. Nessun punto pietà, simonia o premi per la partecipazione o la "buona interpretazione".

\textbf{Non c'è equilibrio}, i giocatori si ritirano se decidono di farno. Non è responsabilità dell'arbitro.

\textbf{Non si bara}; l'arbitro non ha alcun interesse nei risultati. Non c'è mai una ragione per barare. Le regole sono applicate all'aperto e basandosi sulla fiducia reciproca.