\chapter{Introduzione}

Dungeons\&Dragons è un gioco stupendamente profondo con una cultura di gioco lunga e variegata. E sin dai primi inizi, mezzo secolo fa, ci sono stati diversi stili di gioco. Voglio raccontarti di un modo particolare di comprendere e approcciarti al gioco. L'argomento è immensamente vasto, ma voglio essere succinto e non farmi distrarre da divagazioni. Per arrivare rapidamente da qualche parte, abbiamo bisogno di concentrazione. Prima leggi, poi gioca con queste nuove conoscenze, quindi preoccupati di studiare ulteriormente.

Io mi approccio all'argomento da una posizione di possesso, acquisita con la moneta di una pratica autentica. Spero che non di troppo fastidio ad alcuno, se mi metto a riversare le mie opinioni sopra un gioco e un'istituzione così amati. Non oserei farlo se non sentissi una certa dedizione alla materia.

Dirò molte cose in questo libro che contraddiranno direttamente altri testi che parlano di come \dnd debba essere compreso e giocato. In alcuni casi, sarà perché starò onestamente cercando di spiegare meglio lo stesso argomento. In altri, è perché sto presentando uno stile di gioco completamente diverso. E, a volte, i testi precedenti sono semplicemente sbagliati. Cercherò di sbagliarmi meno, ma sentiti libero di correggermi nel prossimo giro di corrispondenza.

\textbf{I wargame} sono giochi di simulazione di un conflitto che hanno le loro radici nel lontano diciannovesimo secolo, prima di qualsiasi cosa assomigliasse a un'industria del gioco. Sono stati tra i primi "giochi di designer", creati da autori noti per bisogni moderni. Mentre altri generi di giochi da tavolo o di carte si focalizzavano sull'elegante interazione di regole formali, i \textit{wargame} di allora scoprirono un mondo diverso: i giocatori potevano creare un \textbf{mondo virtuale} e \textbf{simularne} il funzionamento. Mettendo da parte l'idea di un'identificazione col personaggio e delle storie, i \textit{wargame} sono stati tra i primi giochi di ruolo, nel senso di un gioco che avviene all'interno di un'arena immaginata. Le pedine del gioco in un vero \textit{wargame} sono soltanto le note e le astrazioni della sottostante realtà simulata che è lo scopo principale del gioco.

Gli ideali del \textit{wargaming} erano diversi da quelli degli sport o degli altri giochi da tavolo poiché il \textit{wargaming} era uno strumento di appendimento, insegnamento, dibattito e auto-miglioramento. Il gioco si sarebbe occupato di argomenti che erano vere preoccupazioni umane e la procedura di gioco era pensata per introdurre e produrre informazioni reali. Essenzialmente, il processo di gioco era un processo di investigazione. C'era una sfida, c'erano delle regole e c'era comprensione delle regole, ma rea tutto al servizio di permettere ai giocatori di mantenere il nobile fantasma della simulazione e imparare da esso.

\textbf{\dnd è nato} come un \textit{wargame}, dalla cultura del \textit{wargaming} americano di metà secolo, lanciata nello studio di game design e casa editrice Avalon Hill. Fin dall'inizio, lì si trovavano ogni genere di interessi creativi; non sto sostenendo che questa industria dei \textit{wargame} sia un qualche cifrario perfetto per divinare l'Intento del Creatore. Ma è una lente interessante, che era chiaramente importante per le origini del gioco.

Nelle mie investigazioni dei giochi di ruolo di avventure fantasy sono andato all'indietro nel tempo, dal moderno \dnd del 21° secolo verso un gioco d'avventura fantasy basato sulle sfide. La mia motivazione particolare è stata il cercare di arrivare a un gioco più autentico, meno focalizzato sui contenuti pre-generati e più sui praticanti al tavolo da gioco: meno seguire delle regole create da un saggio lontano, meno Game Master che si comporta da burattinaio. Il gioco di avventura sfidante dovrebbe occuparsi di vittorie e sconfitte vere, realizzazione emergente, imparare gli uni dagli altri, giocare per scoprire cosa accadrà e creare le nostre soluzione.

Scoprire come queste mie recenti preoccupazioni si agganciano sul vecchia cultura da \textit{wargaming} del \dnd iniziale è stato meraviglioso! Stiamo parlando di superare un golfo culturale ampio circa 30 anni. Mi ci è voluto un po' per arrivare effettivamente a sedermi e leggere qualcosa come \textsc{Keep on the Borderlands} (uno dei primi moduli di avventura di \dnd, che ha definito il genere), ma quando l'ho fatto è stato uno di quei momenti da \textit{face-palm}. "Ok, Eero, quindi evidentemente una volta esisteva un gioco di ruolo d'avventura fantasy che investigava esattamente questa cosa, questa specie di sfida dell'avventura, che tu stai cercando di sviluppare. Essenzialmente dimenticata e mal compresa dalla moderna maggioranza, pure. Che ne dici di studiarla un po' di più?"

Questo è successo più o meno attorno a quel periodo, nei tardi '00, in cui l'Old School Reinassance stava prendendo piede, quindi non è che saperne di più fosse difficile. Nei miei diari di gioco, quello che considero il mio primo passo sulla via del \textit{wargaming} con una dottrina pienamente matura avvenne nell'estate del 2011; fu un grande successo creativo locale e la campagna lanciò numerose altre campagne collegate, alcune delle quali stanno ancora proseguendo ad oggi.

Non pretendo che quello che abbiamo scoperto, questa "via del \textit{wargaming}", sia la stessa maniera in cui altri si rapportano al gioco, ma so che è molto funzionale ed è un gioco molto significativo da giocare e che merita un testo introduttivo su come farlo.