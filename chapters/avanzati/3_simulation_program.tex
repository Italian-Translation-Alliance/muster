\usection{Il Programma di Simulazione}

Dungeons \& Dragons è un \wargam{e} che parla di avventura e di narrativa avventurosa. Si versa un sacco di inchiostro su come il gioco sia eccitante ed eroico. Ai tempi, l'attenzione si concentrava su quanto il gioco fosse o non fosse realistico. Si dice fin troppo poco delle assunzioni sottostanti la simulazione del mondo, in particolare visto che \dnd{}, da questo punto di vista, è stranamente ambizioso per un \wargam{e}. Penso che ora possa valer la pena di guardare a che cos'è la cosa effettivamente interessante fatta dal gioco.

\usubsection{Cosa stiamo simulando?}

Il famoso sistema di livelli di personaggio di \dnd{} permette agli avventurieri di diventare più forti man mano che hanno successo nelle avventure, dedicandosi a nuove avventure più pericolose ed eccitanti. Perché? Non è strano che qualcuno diventi "più forte" per merito dell'"andare all'avventura"?

L'idea è spesso presentata come una modellazione dello sviluppo di abilità e l'esperienza nella carriera da parte del personaggio, qualcosa che ha poco senso quando guardi alla direzione generale delle regole. Se le regole simulano addestramento ed esperienza, per come effettivamente funzionano per gli esseri umani, lo fanno veramente male. Gli esseri umani diventano più traumatizzati, non più abili, man mano che vengono usati come sacco da boxe dai mostri.

Un brutto sottotono della questione viene portato dai chiari intenti di gamification introdotti dagli sviluppatori iniziali del gioco; Gary Gygax, per esempio, vide indubbiamente il potenziale di variare premi e penalità in stile gioco d'azzardo e più spesso che no preferiva (cinicamente, secondo me) l'intrattenimento all'illuminazione. Non sempre, l'uomo lottava chiaramente con la natura creativa dell'attività, il gioco sembra prendere fin troppo spesso forma dall'orribile convenienza della tirannia del divertimento. La risposta al perché i personaggi diventano più potenti col passare del tempo è che è il modo migliore di prenderti all'amo, tesoro.

Se questa fosse la miglior risposta alla domanda del programma simulativo di \dnd{}, il gioco non sarebbe meglio dell'infinità dei suoi discendenti videoludici. Una scatola di Skinner meccanica, che insegna ai giocatori a deliziarsi di monete d'oro immaginarie, mentre le distribuisce a velocità ottimizzate per la soddisfazione. Vedi crescere i numeri, una barra di progresso fa "ding!"; l'ottundente oppio della nostra era.

\usubsection{Un modello di narrativa eroica}

Mettendo da parte questa deprimente non-risposta, ecco la sostanza positiva che vedo in \dnd{}: il programma di studi del gioco come \wargam{e} si interessa al livello mutevole di \textbf{privilegio drammatico}: ci si aspetta che i personaggi di livello più alto modellino il modo in cui i protagonisti avventurosi di generi di narrativa d'avventura più ottimisti e stilizzati operino e si rapportino col mondo attorno a loro.

Con questa interpretazione, i personaggi giocanti iniziano al primo livello e avanzano verso i livelli superiori \textit{attraverso il successo nelle avventure} perché l'attività di gioco fa da prova e dimostra la natura di protagonista di un dato personaggio. Il gioco modella il protagonismo attraverso l'idea fantasiosa che possiamo scoprire chi sono gli eroi solo vedendo chi sopravvive ad avventure eroiche. Il tuo personaggio ottiene i vantaggi modellati di essere "l'eroe" come conseguenza dell'essere già sopravvissuto con successo a un'avventura eroica.