\usection{Cose tecniche da fare}

Mi rendo conto che molto di quello che sto scrivendo in questo libro è molto generico, molto astratto. Ci sono buoni motivi, come il fatto che cerchi di concentrarmi sullo spiegare correttamente le cose importanti, invece di preoccuparmi di ciò che è mutevole, delle questioni di gusto di ciò che dipende da un contesto generale più ampio. C'è un equilibrio in questo e almeno alcuni lettori troverebbero tutto questo più facile da seguire se il contesto fosse più concreto. Ma per questo ci sono i regolamenti, tu stai leggendo solo un umile commentario.

Quindi proviamo così. Prenderò uno dei regolamenti suggeriti prima - non importa quale - e sceglierò alcuni punti tecnici basilari che potrebbero essere svolti in un modo o nell'altro e ti dirò come lo faccio io e perché.

\usubsection{Punteggi di Caratteristiche}

Il metodo 3d6-in-ordine funziona meglio per questo gioco, non credere ai vari tiri modificati o agli schemi di point buy. Questi vanno bene esclusivamente per giochi dove i personaggi sono persone importanti di default. In una campagna di \wargam{ing} ad alta letalità, è meglio se la generazione del personaggio è veloce, i personaggi non sono eccezionali e l'occasionali lista di caratteristiche eccezionalmente buona è causa di eccitazione.

Se siete come me, allora la grossa questione filosofica con le caratteristiche veramente casuali è lo sbilanciamento permanente e il bruciante desiderio di giocare un personaggio con delle buone caratteristiche. Semplicemente, non è molto divertente iniziare in un modo permanentemente azzoppato e su cui non puoi fare niente, solo perché ti è capitato di fare brutti tiri. Diventa in qualche modo pure peggio se per qualche motivo è questo il personaggio che arriva al 2° livello, perché stai costruendo un eroe su fondamenta difettose. Mi immagino che questo tipo di ragionamento sia stato una grossa motivazione dietro al fatto che, negli anni, \dnd{} abbia progressivamente abbandonato l'altrimenti superiore generazione casuale delle caratteristiche.

La regola d'oro che risolve questo singolo difetto è piuttosto ovvia: se i punteggi di caratteristica possono cambiare durante il gioco (qualcosa che i regolamenti ortodossi tendono a scoraggiare per cattive ragioni), coi personaggi che guadagnano e persino perdono punti durante le loro avventurose carriere, la tirannia iniziale della fortuna diventa nuovamente appetibile. 

L'arbitro furbo può persino usare diverse regole di addestramento per incoraggiare gli avventurieri a spendere soldi e tempo per risolvere serissimi difetti nella propria costituzione. Ti piacciono i money sink, giusto?

\textbf{In breve:} La generazione dei punteggi di caratteristica dovrebbe inizialmente essere molto rigida. Dovrebbe, tuttavia, essere possibile migliorare le statistiche iniziali col progredire sul lungo termine della campagna. I punteggi di caratteristica non dovrebbero essere un così gran problema per personaggi di alto livello!