\usection{Cose tecniche da fare}

Mi rendo conto che molto di quello che sto scrivendo in questo libro è molto generico, molto astratto. Ci sono buoni motivi, come il fatto che cerchi di concentrarmi sullo spiegare correttamente le cose importanti, invece di preoccuparmi di ciò che è mutevole, delle questioni di gusto di ciò che dipende da un contesto generale più ampio. C'è un equilibrio in questo e almeno alcuni lettori troverebbero tutto questo più facile da seguire se il contesto fosse più concreto. Ma per questo ci sono i regolamenti, tu stai leggendo solo un umile commentario.

Quindi proviamo così. Prenderò uno dei regolamenti suggeriti prima - non importa quale - e sceglierò alcuni punti tecnici basilari che potrebbero essere svolti in un modo o nell'altro e ti dirò come lo faccio io e perché.

\usubsection{Punteggi di Caratteristiche}

Il metodo 3d6-in-ordine funziona meglio per questo gioco, non credere ai vari tiri modificati o agli schemi di point buy. Questi vanno bene esclusivamente per giochi dove i personaggi sono persone importanti di default. In una campagna di \wargam{ing} ad alta letalità, è meglio se la generazione del personaggio è veloce, i personaggi non sono eccezionali e l'occasionali lista di caratteristiche eccezionalmente buona è causa di eccitazione.

Se siete come me, allora la grossa questione filosofica con le caratteristiche veramente casuali è lo sbilanciamento permanente e il bruciante desiderio di giocare un personaggio con delle buone caratteristiche. Semplicemente, non è molto divertente iniziare in un modo permanentemente azzoppato e su cui non puoi fare niente, solo perché ti è capitato di fare brutti tiri. Diventa in qualche modo pure peggio se per qualche motivo è questo il personaggio che arriva al 2° livello, perché stai costruendo un eroe su fondamenta difettose. Mi immagino che questo tipo di ragionamento sia stato una grossa motivazione dietro al fatto che, negli anni, \dnd{} abbia progressivamente abbandonato l'altrimenti superiore generazione casuale delle caratteristiche.

La regola d'oro che risolve questo singolo difetto è piuttosto ovvia: se i punteggi di caratteristica possono cambiare durante il gioco (qualcosa che i regolamenti ortodossi tendono a scoraggiare per cattive ragioni), coi personaggi che guadagnano e persino perdono punti durante le loro avventurose carriere, la tirannia iniziale della fortuna diventa nuovamente appetibile. 

L'arbitro furbo può persino usare diverse regole di addestramento per incoraggiare gli avventurieri a spendere soldi e tempo per risolvere serissimi difetti nella propria costituzione. Ti piacciono i money sink, giusto?

\textbf{In breve:} La generazione dei punteggi di caratteristica dovrebbe inizialmente essere molto rigida. Dovrebbe, tuttavia, essere possibile migliorare le statistiche iniziali col progredire sul lungo termine della campagna. I punteggi di caratteristica non dovrebbero essere un così gran problema per personaggi di alto livello!

\usubsection{Classe e Livello}

Le classi dei personaggi sono l'espressione pratica del concetto di eroismo nella campagna. Questo vuol dire che le classi disponibili per i personaggi dovrebbero dipendere dall'ambientazione e il focus tematico centrale della campagna. A questo riguardo non ci sono scelte sbagliate, in particolare quando capisci che la classe di un personaggio può essere definita e descritta in modo ampio e astratto, oppure in modo estremamente preciso e concreto, e queste possibilità servono uno scopo.

I personaggi di 1° livello dovrebbero essere considerati solo incrementalmente più potenti, esperti, saggi, eroici e fighi rispetto a una persona qualunque nell'ambientazione. È solo con l'aumento di livello che diventano più eccezionali; il 2° livello di un personaggio è essere il capo di una banda, e il 3° livello è essere il protagonista di un film d'avventura molto crudo. Alcuni testi di \dnd{} sono diventati sempre più confusi riguardo a questo nel corso del tempo, in entrambe le direzioni, quindi ripagherà ripassare i presupposti.

Le regole di \dnd{} sin dalla pubblicazione di A\dnd{} hanno avuto una spiccata tendenza a strutturare "il gioco a lungo termine" tramite livelli con valori a due cifre. Raccomando però di rimanere più sui livelli 1-10, i "bassi livelli"; la matematica del gioco generalmente funziona in modo più naturale, e non ti impedisce di avere dell' \textit{high fantasy} nel tuo gioco. Così, al contrario, l'arbitro non può giocare a rimpiattino definendo tutte le cose veramente belle dell'ambientazione come esistenti solo a livelli irraggiungibili.

\textbf{In breve:} Le classi dovrebbero riflettere l'ambientazione e i livelli dovrebbero modellare coerentemente i gradi di eroismo.

\usubsection{Punti Esperienza}

È di assoluta importanza che le regole degli XP siano orientati sugli obbiettivi, basati sul raggiungimento di traguardi e senza bias, per ragioni che discuteremo ampiamente in questo libro.

Oltre a questo, raccomando di coltivare schemi di conferimento dell'esperienza consistenti, trasparenti e ben definiti che si allineano con gli obbiettivi dello scenario: la cosa che i giocatori cercano di fare nello scenario è la cosa che gli fa guadagnare punti, per definizione.

Inoltre consiglio di non far scalare gli XP direttamente o indirettamente sul livello del personaggio. I personaggi che vogliono ottenere più XP dovrebbero entrare in un nuovo livello di gioco con ricompense in XP differenti, non aspettarsi del tesoro garantito dal GM per un ritmo di avanzamento adeguato.

Raccomando fortemente di praticare \textbf{il limite del guadagno XP}, una regola che limita un avventuriero a guadagnare al massimo il tetto necessario di XP per passare di livello in un singolo scenario. Aiuta a gestire l'economia degli XP in caso di eventi inaspettati.

Dato che gli XP sono legati all'avanzamento degli avventurieri nelle proprie classi, raccomando una \textbf{progressione esponenziale} pura: nelle vecchie regole di \dnd{} è quella cosa secondo la quale il totale di XP richiesto per il livello successivo è sempre il doppio dell'ammontare richiesto per il livello precedente. Questa configurazione funziona molto bene a fianco a tecniche di gestione della campagna ottimali come scuderie di personaggi, gruppi di livello variabile e la ripartenza dal 1° livello dopo la sconfitta.

\textbf{In breve:}  Tratta i punti esperienza similemente ai punteggi di un videgioco. Oggettivi, consistenti e trasparenti.

\usubsection{Punti ferita e relativo aumento}

È uno delle meccaniche estetiche principali di \dnd{} che ottieni un dado di punti vita, un Dado Vita, per livello del personaggi. E i grossi mostri non umani possono avere anche loro Dadi Vita.

Anche se le tue regole non usano i dadi vita, mi concentrerei piuttosto nell \textbf{aumento lineare dei punti vita} per livello: 1 unità di essi al 1° livello, 2 al 2° e così via. È una concezione così centrale del come il gioco usa i punti ferita per simulare le questioni narrative dell'avventura. È la base dell'intera simulazione.

Poi, come corollario, \textbf{l'aumento logaritmico del danno}. Se applichi [questo ammontare di danno] x [livello] assieme ai punti ferita lineari, non stai permettendo una relazione proporzionale fra pericolo e avventurieri per svilupparsi e cambiare man mano che i livelli salgono. Invece della scala lineare dei pool dei punti ferita, il potenziale di danno dovrebbe essere 1d6 al 1° livello e 3d6 al 10° livello.

Questo è un fraintendimento comune nelle gestione delle regole più moderne di \dnd{}, un'insistenza nello scalare tutti i numeri in egual modo. Il problema con l'aumento uniforma è che genera un progresso illusorio e distacco dalla simulazione; le cose \textit{devono} cambiare mentre sali con i livelli!

\textbf{In breve:} I punti ferita scalano linearmente per livello, mentre i numeri del danno aumentano solo lentamemte, risultando in una tendenza generale verso sequenze più lunghe di azioni e combattimento agli alti livelli.
