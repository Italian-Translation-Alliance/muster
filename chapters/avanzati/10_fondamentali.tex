\usection{Gli Elementi Fondamentali dell'Arbitraggio}

I testi regolistici di \dnd{} contengono, fin dagli inizi, un errore metodologico intrattabile che i giocatori esperti sanno come compensare, ma che inganna di continuo i nuovi arrivati. L'errore sta nel fatto che le grandi costellazioni di regole lì presenti sono presentate come \textit{importanti}: per giocare, è imperativo che tu studi queste regole e che le applichi perfettamente. E, se il gioco non è piacevole come dovrebbe, la colpa è o tua per non averle applicate correttamente, o delle regole per non essere le Regole Perfette, quindi è meglio se vai a comprarne ancora, nella speranza di arrivare al vero gioco. Si tratta di un'istituzione culturale rotta.

Alcuni giochi di ruolo operano effettivamente partendo da un immutabile telaio sistemico e questi giochi hanno solitamente delle buone regole, ma \dnd{} nella maniera del \wargam{ing}, esplicitamente, non lo fa: la \textbf{priorità simulativa} del gioco significa apertamente che, ove la tua comprensione delle circostanze fittizie ti porta a una conclusione diversa da quella delle regole, le regole cedono il passo. La corretta e propria risoluzione dello scenario ha una precedenza metodologica sopra qualsiasi cosa possa star scritta un qualche stupido regolamento.

Il grande Matt Finch ha enunciato questo principio come "\textit{Rulings, Not Rules}"\translatornote{"Decisioni, non regole", ma è un'espressione talmente diffusa nella cultura di gioco che ho preferito tenerla com'era nel testo principale.}, ovvero che la procedura di risoluzione effettivamente utilizzata in gioco non è la mera applicazione di una regola scritta su un libro. Piuttosto, è compito dell'arbitro considerare la simulazione nella sua interezza e di prendere una \textbf{decisione} equa sulla situazione corrente.

Regole e decisioni intelligenti sono il cuore del medium del gioco simulativo, gli elementi che usi per costruire l'interfaccia con cui i giocatori toccheranno gli spazi virtuali e immaginati. Secondo la via del \wargam{ing}, queste interfacce derivano dalla situazione narrativa: l'Ogre è inumanamente forte \textit{e quindi} ottiene un bonus alle prove di forza.

Il compito dell'arbitro è di guidare l'assalto del gruppo contro il compito immensamente complesso che è una simulazione corretta e intelligente; il mondo non è forse un luogo complesso e caotico? Il compito non è mai eseguito alla perfezione, ma puoi essere soddisfatto dell'averlo svolto bene date le circostanze e con l'ottenere una simulazione coerente che affronti i grandi problemi dello scenario. Se un gioco si concentra sull'uso di schermi di cavalleria per confondere gli esploratori nemici, allora ottenere una costellazione di regole e decisioni che catturi i particolari del funzionamento di questa pratica è già abbastanza.

\usubsection{Le Basi}

Un arbitro di \dnd{} potrebbe voler fare un giro del cosiddetto \textbf{free kriegsspiel} per farsi un'idea della transazione fondamentale a cui puntano tutte le regole di un \wargam{e}. Personalmente, ho trovato una spiegazione chiara e persuasiva del fuzionamento dei \textit{free kriegsspiel} nei giochi \textsc{Engle Matrix} di Chris Engle, ma i buoni testi introduttivi sull'argomento diventano ogni giorno più comuni.

L'atto pratico del gioco \textit{free kriegsspiel} è essenzialmente lo stesso di \dnd{}, tranne per il catto che non si comincia con le cornici regolistiche complesse e dettagliate di \dnd{}. Invece, l'unica regola stabile è regola che stima le probabilità: un giocatore esegue una manovra, un'azione che affronti la sfida dello scenario, e l'arbitro decide quanto è probabile che la manovra descritta abbia successo. Questa decisione è quindi portata avanti con un tiro di dado e se ne valutano le conseguenze.

Per alcuni arbitri di \dnd{}, potrebbe risultare illuminante studiare come il gioco si svolga senza classi di personaggio, punti ferita, regole per l'esperienza e turni di combattimento - senza la costellazione che oscura quello che stai effettivamente facendo quando giochi al gioco. La conclusione dovrebbe essere, ritengo, che sia comunque lo stesso gioco, una specie di \dnd{} primitivo e ridotto alla sua struttura ultima.

Tornando dal \textit{free kriegsspiel} a \dnd{}, nulla cambia davvero: un avventuriero desidera usare una corda per attraversare un baratro. Non mi stupirei davanti a una campagna di \dnd{} dove l'arbitro decide direttamente che questa è un'operazione piuttosto pericolosa anche quando eseguita con cura, quindi non cadi in 4-su-6, tiriamo per vedere cosa succede. Non chiedere il punteggio di Destrezza del personaggio, puntando al concetto fondante di probabilità già valutata, sarebbe piuttosto primitivista, ma non particolarmente illegittimo.

Un arbitro con un'inclinazione leggermente maggiore ad usare davvero le costellazioni meccaniche di \dnd{} userebbe certamente un qualche tipo di prova di Destrezza; fai meno del punteggio di DEX su un d20 o prendi un bonus alla prova col d6 di cui sopra se hai un bonus di DEX o qualcos'altro. Magari persino una prova di Forza. Se esiste un sistema di abilità, sicuramente avrà un ruolo importante. Una campagna con una comprensione piuttosto metafisica del concetto di Livello, o una con una caratteristica di Fortuna o dei Tiri Salvezza, potrebbe portare quel tipo di eroismo ineffabile in gioco quando ci si trova davanti a un pericolo mortale.

È per questo che dovresti studiare i regolamenti, per avere sotto mano questi strumenti mentre crei costellazioni meccaniche piacevoli per la tua campagna, o per utilizzarli quando avviene una situazione inaspettata in gioco.

\usubsection{Decisioni, non regolamenti}

Comunque, le regole non coprono ogni situazione, né dovrebbero farlo. È già stato provato e non funziona per un gioco che permetta vera libertà di manovra; la risoluzione dei problemi da parte dei giocatori viene soffocata da una montagna di testi regolistici in continua crescita e i formalismi meccanici si dimostrano sempre più importanti del problem solving autentico e situazionale. Il \dnd{} troppo dipendente dalle regole diventa un edificio turgido in cui un'incantesimo è sempre meglio di un'idea, semplicemente perché l'incantesimo è scritto in qualche regolamento, pronto per avere un'interazione potente con altre regole, mentre l'idea non ha altro che i giocatori dalla sua. Se questi giocatori non vogliono o non sono capaci di difendere le loro idee dai regolamenti, allora i regolamenti vinceranno ogni volta.

Così impedito a poter rispondere a tutto con delle regole, l'arbitro è costretto a fare affidamento sulle decisioni, come è giusto che sia. Prendere queste decisioni può essere spaventoso, ma la procedura del \textit{free kriegsspiel} è interamente sufficiente come base: sarai un ottimo arbitro se semplicemente sarai disposto a decidere un numero. Quanto è possibile che questa manvora funzioni, espresso in unità percentuali o sulle facce di un dado? Hai sicuramente qualche idea. È 20\% possibile? 80\%? Una qualche via di mezzo?

Vedi, il processo di gioco non richiede che la tua stima sia corretta in senso assoluto, ma solamente che tu prenda una decisione e vada avanti. Il momento unitario in cui si determina una soluzione usando i dadi riflette un evento unico nella storia del mondo di gioco; può avere un peso come precedente più in là nella campagna, ma non è tenuto a farlo. Hai tempo di riconsiderare più avanti e migliorare mentre giochi.

\usubsection{Alcuni aspetti pratici}

Ciò che viene effettivamente fatto dall'arbitro mentre "si simula" in questo tipo di giochi è piuttosto intricato e sfortunatamente glissato dalla maggior parti dei testi dei regolamenti. Darò giusto qualche indicazione pratica per dare un senso dell'argomento.

Innanzitutto, impara a risolvere eventi atomici come discusso sopra. Una piccola formalizzazione che trovo utile e mi piace usare è data dalle rozze categorie di eventi "probabili" e "improbabili", che hanno, rispettivamente, il 50\% (3-su-6) e il 15\% (1-su-6) di possibilità di verificarsi. Considerando la maggior parte degli eventi atomici, che siano attività o eventi casuali, probabilmente puoi decidere se sono più probabili che improbabili. Ovviamente, puoi anche invertire le chance di "improbabile" per ottenere "estremamente probabile" a 5-su-6.

In secondo luogo, impara ad atomizzare la simulazione di eventi complessi. Potresti non avere idea di quanto sia probabile un evento complesso nella sua interezza, ma se riesci a spezzarlo in una sequenza causale di precondizioni necessarie che sono, singolarmente, più semplici da valutare, potresti riuscire ad arrangiarti. Ho notato che fare una lista di "sfide" è un approccio utile in questo senso: fai una lista delle difficoltà e dei problemi che emergono in una sequenza più complessa e poi affrontali uno alla volta col gruppo, capendo quanto sia un problema quel determinato aspetto, cosa possono fare i giocatori per affrontarlo e se servano o meno tiri di dado. In effetti, questo è ciò che è giocare a questo gioco: elaborare la simulazione.

Per esempio, quando devi risolvere l'attaversamento di un deserto da parte di una spedizione, ricostruisci se è possibile e sicuro, potrebbe essere possibile fare ricorso a una procedura con un singolo tiro e una generica probabilità probabile/improbabile. Oppure, potresti decidere di affinare le precondizioni necessari, come la conoscenza della strada, l'accumulo di abbastanza acqua e la rara, seppur devastante, eventualità di una tempesta nel deserto, risolvendo ciascuno di questi punti, e altri ancora, uno alla volta, per dipingere un dipinto più dettagliato di quello che succede nell'evento.

Oppure, potresti fare ricorso all'ancora più elaborato, ben rodato, patrimonio di meccaniche che hai a portata di mano. Ci sono costellazioni di regole per qualsiasi cosa là fuori, e tu hai anche le tue. \dnd{} sta alla fine del viaggio del \textit{free kriegsspiel}, non al suo inizio. Eppure, io finisco a prendere decisioni primitive e a usare tiri di probabilità atomiche ogni volta che giochiamo.