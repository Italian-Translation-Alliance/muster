\usection{Gli Elementi Fondamentali dell'Arbitraggio}

I testi regolistici di \dnd{} contengono, fin dagli inizi, un errore metodologico intrattabile che i giocatori esperti sanno come compensare, ma che inganna di continuo i nuovi arrivati. L'errore sta nel fatto che le grandi costellazioni di regole lì presenti sono presentate come \textit{importanti}: per giocare, è imperativo che tu studi queste regole e che le applichi perfettamente. E, se il gioco non è piacevole come dovrebbe, la colpa è o tua per non averle applicate correttamente, o delle regole per non essere le Regole Perfette, quindi è meglio se vai a comprarne ancora, nella speranza di arrivare al vero gioco. Si tratta di un'istituzione culturale rotta.

Alcuni giochi di ruolo operano effettivamente partendo da un immutabile telaio sistemico e questi giochi hanno solitamente delle buone regole, ma \dnd{} nella maniera del \wargam{ing}, esplicitamente, non lo fa: la \textbf{priorità simulativa} del gioco significa apertamente che, ove la tua comprensione delle circostanze fittizie ti porta a una conclusione diversa da quella delle regole, le regole cedono il passo. La corretta e propria risoluzione dello scenario ha una precedenza metodologica sopra qualsiasi cosa possa star scritta un qualche stupido regolamento.

Il grande Matt Finch ha enunciato questo principio come "\textit{Rulings, Not Rules}"\translatornote{"Decisioni, non regole", ma è un'espressione talmente diffusa nella cultura di gioco che ho preferito tenerla com'era nel testo principale.}, ovvero che la procedura di risoluzione effettivamente utilizzata in gioco non è la mera applicazione di una regola scritta su un libro. Piuttosto, è compito dell'arbitro considerare la simulazione nella sua interezza e di prendere una \textbf{decisione} equa sulla situazione corrente.

Regole e decisioni intelligenti sono il cuore del medium del gioco simulativo, gli elementi che usi per costruire l'interfaccia con cui i giocatori toccheranno gli spazi virtuali e immaginati. Secondo la via del \wargam{ing}, queste interfacce derivano dalla situazione narrativa: l'Ogre è inumanamente forte \textit{e quindi} ottiene un bonus alle prove di forza.

Il compito dell'arbitro è di guidare l'assalto del gruppo contro il compito immensamente complesso che è una simulazione corretta e intelligente; il mondo non è forse un luogo complesso e caotico? Il compito non è mai eseguito alla perfezione, ma puoi essere soddisfatto dell'averlo svolto bene date le circostanze e con l'ottenere una simulazione coerente che affronti i grandi problemi dello scenario. Se un gioco si concentra sull'uso di schermi di cavalleria per confondere gli esploratori nemici, allora ottenere una costellazione di regole e decisioni che catturi i particolari del funzionamento di questa pratica è già abbastanza.

\usubsection{Le Basi}

Un arbitro di \dnd{} potrebbe voler fare un giro del cosiddetto \textbf{free kriegsspiel}\translatornote{} per farsi un'idea della transazione fondamentale a cui puntano tutte le regole di un \wargam{e}.