\usection{Il Necromante}

La nostra campagna originale di fantasy storico iniziò nel 2011 e proseguì per cento sessioni (una cifra stranamente tonda) prima che diversi problemi della vita reale spezzassero il gruppo. Una delle avventure più memorabili di quel periodo fu la saga di Voldemort il Necromante, iniziata attorno alla sessione \#66 e proseguita per circa una dozzina di sessioni. Pensavo, all'epoca, ma lo faccio ancora adesso, che fosse un buon esempio di come appaiono gli sviluppi organici di una campagna \textit{sandbox}. Invece di una serie di moduli di avventura affrontati uno alla volta, è più un complesso intreccio di problemi emergenti.

Descriverò brevemente come la saga di Voldemort è nata e si è conclusa. Tutto questo è stato diligentemente fatto in maniera da \wargam{e}, quindi anche se gli eventi possono di tanto in tanto sembrare drammatizzati, ci si è arrivati attraverso la modellazione della situazione e scelte di gioco dinamiche; gli unici piani coinvolti sono stati l'avvio della campagna, le manovre degli avventurieri e le contromanovre dell'opposizione.

Ovviamente, ci sono un sacco di dettagli in una singola sessione di gioco, quindi sarà necessario concentrarci soprattutto sulla struttura generale, piuttosto che sulle manovre da momento a momento.

\textbf{Sessione \#66}: Il gruppo di avventurieri non aveva un obiettivo dopo aver terminato l'ultima avventura; Rolf il Cappellano (Chierico 1), cercando di fare del bene nel vescovado locale, aveva sentito voci preoccupanti di venerazione necromantica in un villaggio remoto. Si trattava di \textsc{Insidious}\translatornote{Di questa avventura non ho trovato alcuna traccia, suppongo sia un modulo scritto da Eero stesso.}, un modulo di avventura di basso livello e piuttosto semplice che stavo per introdurre.

La prima sessione è consistita principalmente di manovre preliminari, viaggiare verso il posto, interrogare popolani e scoprire la magione abbandonata dell'ultimo borgomastro. C'è stato un incontro con delle strigi.

\textbf{\#67}: Un paio dei giocatori principali mi sono venuti a trovare qualche volta nel corso della settimana, per preparare l'effettiva sessione con un po' di esplorazione a bassa posta e mappatura delle parti meno interessanti della magione. Il gruppo ha iniziato la sessione effettiva ben preparato per un'azione immediata.

In quel periodo il gruppo aveva un combattente di 3° livello, una specie di ranger di 2° e un chierico di 4° in aggiunta ai personaggi di 1°, quindi sbarazzarsi degli orchi non è stato, per lo più, un gran lavoro. Incontrare il Necromante di 10° livello nascosto sotto la magione si rivelò una sorpresa inaspettata e sgradita!

Il Combattente del gruppo venne inizialmente \textit{Charmato}, cosa che fu solo l'inizio delle difficoltà; il secondo assalto alla tana del Necromante diventò quasi un \textit{total party kill}, evitato solamente dalla presenza del chierico combattente Juanita e dalla sua forza magica. Il gruppo aveva portato un Paladino Elfo di 3° livello per rimpiazzare il Combattente \textit{charmato}; Varaniel l'Elfo perse i suoi poteri da paladino, non avendo saputo impedire a Juanita di picchiare a morte il Necromante a mani nude.

Successivamente, il gruppo scoprì il libro degli incantesimi del Necromante, un grande tesoro per un gruppo emergente di medio livello che sta investigando i misteri dell'occulto. Scoprimmo che il nome del Necromante era "Voldemort".

Significativamente, questo è stato il primo caso in cui le regole della campagna avevano prodotto una reliquia dalla morte di un personaggio con livelli; Voldemort si lasciò dietro un pendente a forma di teschio che Varaniel l'Elfo presé per sé dopo l'omicidio.

\textbf{\#68}: Abbiamo fatto una sessione infrasettimanale con Timo, il giocatore di Varaniel l'Elfo, stabilendo che Varaniel stesse effettivamente impiegando il pendente-reliquia lasciato dal Necromante per la sua ricerca di incantesimo. Varaniel non era sempre stato un Elfo (aveva precedentemente fatto ricorso alla trasmigrazione dell'anima per scappare da una malattia maledetta), quindi, sebbene possedesse la Potenza Eladrin, sapeva a malapena usare il proprio potenziale magico.

Avevo tirato per determinare la natura della reliquia di Voldermort e, ovviamente, il fato aveva voluto che fosse un filatterio, o un Horcrux, come lo chiamano i ragazzi; Timo era ben documentato su Harry Potter quindi sì, eravamo sulla stessa pagina.

Abbiamo modellato l'influenza che l'Horcrux del Necromante avrebbe avuto sulle azioni di Varaniel nel \textit{downtime} tra le avventure, stabilendo che venisse presto ispirato a fuggire nella notte con il suo nuovo e misterioso servitore "Grego" il Ratto Mannaro (questo tizio era semplicemente apparso senza venir invitato qualche settimana dopo la morte del Necromante, uno sviluppo completamente normale).

Varaniel stesso era convinto di star cercando una scorciatoia verso il potere magico. Noi giocatori sapevamo che stava venendo sottilmente influenzato dal pendente e dai consigli di Grego. Varaniel si trovò presto impegnato ad assoldare guide e a viaggiare nelle terre selvagge, cercando niente meno che il temuto luogo di culto senza nome di \textsc{Death Frost Doom}.

\textbf{\#69}: Una sessione dal passo lento, in seguito ai continui tentantivi del gruppo di trovare e risolvere avventure "fegatelli". Né Timo (il giocatore di Varaniel) né Peitsa (il giocatore di Juanita) erano dei nostri, e l'avventuriero di punta della campagna, Hans Krüger non era ancora pronto per iniziare la sua campagna italiana, quindi ci voleva qualcosa di piccolo.

Alla fine, il gruppo finì per battere una vecchia pista (per qualche motivo pensavano che i documenti di pianificazione di una banda di fuorilegge fossero ancora informazioni correnti dopo 15) e incontrando per caso alcune caverne naturali. La cosa più notevole della sessione fu uando uno degli avventurieri finì davanti a una corte marziale del resto del gruppo e venne condannato a morte per perfida codardia.

\textbf{\#70}: Un'altra sessione in solitaria con Varaniel tra le sessioni regolari. Varaniel e il suo fidato servitore Gregor erano piuttosto soddisfatti di "aver trovato" l'antico altare Duvan'Ku che stavano cercando tra le montagne. Gregor ovviamente, serve Voldemort il Necromante e, di nascosto da Varaniel, sta lavorando per facilitare l'imminente ritorno di detto spettabile personaggio.

\textbf{\#71}: Timo era nuovamente assente (per motivi organizzativi, l'ispirazione parziale all'origine di queste sessioni in solitaria), ma avevamo il resto del gruppo centrale. Avevo abbozzato un dungeon da una pagina piuttosto stupido sul viaggio nel tempo e gli uomini delle caverne che scoprivano il fuoco e, con i giocatori interessati a un completo cambio di passo, facemmo quello per una sessione. I giocatori più furbi non avevano mandato i loro personaggi migliori e, infatti, nessuno dei quattro coraggiosi mandati nel passato è tornato indietro.

Alla fine della sessione Juanita, un personaggio centrale nei tentantivi del gruppo di costruire una base, finalmente scoprì che Varaniel l'Elfo era scomparso; che se ne andasse in un misterioso viaggio privato non era così strano, ma quando Juanita si è accorto anche della mancanza del libro degli incantesimi di Voldemort è \textit{assolutamente} riuscito a unire i puntini!

\textbf{\#72}: Una terza sessione solitaria per Varaniel l'Elfo. Varaniel è sceso nelle fauci del luogo di culto senza nome. Fu molto coraggioso, forse un po' confuso, e immensamente fiducioso nel fatto che la sua carriera magica stesse per fiorire. Gregor lo scagnozzo lo incoraggiava, ma senza tradire ulteriore conoscenza di questo luogo malvagio.

Varaniel seguì il sussuro senza esitare e, infatti, la testa conservata di Voldemort, assieme al suo grimorio e all'amuleto Horcrux, permise alla coppia di entrare nel \textit{sancta sanctorum} del tempio.

Timo si aspettava assolutamente che Gregor lo tradisse nel momento chiave, e non mi aspettavo che "facesse lo stupido" in alcun modo, ma il tempismo dell'assalto lo colse alla sprovvista e la maledizione magica dell'altare maggiore colse Varaniel con la guardia abbassata. Gregor il Ratto Mannaro sapeva che uno dei due sarebbe dovuto morire perché Voldemort vivesse di nuovo, e non intendeva essere lui.

Il difetto principale di Varaniel, come avventuriero ed eroe, erano, fin da quando subì una maledizione divina per aver derubato un tempio nella sessione \#8, le sue terribili statistiche fisiche. Non fu in grado di opporre resistenza al Ratto Mannaro e impedire di venir sacrificato su quell'altare terribile!

\textbf{\#73}: Il gruppo aveva scoperto che Varaniel li aveva traditi o, potenzialmente, era stato incantanto (la giuria era ancora indecisa su quest'idea - la stessa giuria che aveva condannato a more Vendetta numero-qualcosa, uno dei loro, poche sessioni prima) alla fine della sessione precedente. Gli avventurieri erano estremamente risentiti, i giocatori eccitati all'improvvisa spinta di manovre giocatore-contro-giocatore. Timo creò un nuovo personaggio per unirsi alla caccia, non volendo portare il suo combattente, ancora \textit{charmato}.

I giocatori erano all'oscuro di quanto era accaduto e io e Timo, ovviamente, non gli abbiamo rivelato nulla. La sessione si è evoluta in una caccia al cardiopalma mentre il gruppo viaggiava dalla fantasy-Baviera alla fantasy-Olanda, cercando la fortunatamente facile da riconoscere faccia Eladrin di Varaniel. Diversi indizi lasciati durante il suo passaggio hanno aiutato il gruppo a trovare la strada per \textsc{Death Frost Doom}.

Il party scalò la montagna, esplorando le caratteristiche superficiali, e ebbe il buonsenso necessario a verificare che una coppia che ricordava quella descritta da loro avesse già lasciato la montagna due giorni prima. A questo punto, i giocatori fecero la scelta molto intelligente di non provocare ulteriormente \textsc{DFD} e di continuare invece a seguire Varaniel e quel grimorio pericoloso e blasfemo.

L'inseguimento, con un'azione serrata, attraverso la fantasy-Olanda, arrivando finalmente al suo apice ad Amsterdam, dove il gruppo invididuò Varaniel-e-Voldemort giusto un giorno prima che la sua nave lasciasse il porto per terre lontane. Sfortunatemente, tre dei cinque avventurieri si erano distratti ad Amsterdam (fondamentalmente avevano aperto la loro fiera per qualche motivo?), il che significava che Juanita e Yoreel (Ranger 3) finirono per mettere all'angolo il Necromante rinato da soli, in svantaggio.