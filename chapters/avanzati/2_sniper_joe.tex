\usection{La Saga di Joe il Cecchino}

Lascia che ti parli del mio personaggio, potrebbe rivelarsi illustrativo per quanto riguarda gli archi di esperienza che i giocatori e i loro personaggi ammassano nel tempo.

Tendo ad arbitrare molto più spesso di quanto non giochi, ma a metà degli anni '10 ho avuto la possibilità di partecipare come giocatore durante un consideravole arco di campagna. Heikki stava gestendo una campagna fantasy-storica tutta sua a quel punto, una specie di discendente della mia partita di prima. L'avventura che Heikki stava portando era \textsc{Rahasia}, un modulo classico della TSR, ma, siccome la campagna era un sandbox dove avevamo accesso alla panopilia completa di manovre strategiche, l'avventura stava lentamente diventando qualcosa di irriconoscibile, mentre avventurieri e streghe si combattevano nella Moldavia medievale.

Arrivai circa a metà scenario e persi ignominosamente il mio primo personaggio in un incontro in un dungeon. Il secondo, invece, durò più a lungo; "Joe il Cecchino", il mio Combattente (a dire il vero era un Ranger, ma alle regole di LotFP non importa), finì per giocare un ruolo decisivo nel nostro salvataggio del regno dalle minacce dell'antica stregoneria.

Avevo battezzato Cecchino come uno degli \textit{sprite} nemici dei vecchi giochi di \textsc{Megaman}\translatornote{Non mi risulta alcuna traduzione italiana dei nomi degli \textit{sprite} di Megaman, tutto quello che posso suggerirvi è di cercare su Google per vedere se li riconoscete. Io no.} (cosa che suppongo mi dati), l'identità del personaggio era a dire il vero lo stesso tipo di commedia tragica di cui riempio i miei giochi: Joe era un "veterano del Vietnam" (eravamo implicitamente d'accordo che sarebbe stato un veterano di una qualsiasi guerra del recente passato dell'ambientazione, era una similitudine più che un fatto letterale), un tizio taciturno e paranoide che aveva trasformato il proprio PTSD in un'autodistruttiva seconda carriera come avventuriero. Si trattava semplicemente di un po' di leggero riempitivo narrativo per divertirmi mentre la situazione si sviluppava; sono il tipo di giocatore a cui piace stabilire ed immaginare un po' dell'identità del personaggio.

Cecchino si rese piuttosto utile nel dungeon delle streghe, ma il suo vero momento di gloria arrivò quando il gruppo rientrò in città dopo un'escursione incompleta nel dungeon per scoprire che le streghe stavano tentando un golpe l'intera baronia con \textit{Charme su Persone}. La città era un vespaio di notizie sull'improvvisa storia d'amore tra il barone e una ragazzina paesana senza nome, arrivata inappropriatamente vicina alla vedovanza di quest'ultimo; il gruppo riuscì a mettere insieme i pezzi abbastanza rapidamente, realizzando che le streghe scambia-corpi erano emigrate dal dungeon alla città, mettendo a rischio la struttura stessa della società!

La paranoia di Cecchino ingranò di brutto quando iniziammo a sospettare della situazione, così mi lanciai completamente in modalità spionaggio per infiltrare il gruppo in città, identificare la cabala delle streghe e poi organizzare la loro eliminazione a fronte della lealtà (alimentata a \textit{Charme}) del Barone verso la sua nuova moglie. A posteriori, avevo principalmente sprecato tempo, visto che le streghe non si rivelarono chissà che geni del male, ma quantomeno il gruppo riuscì ad operare senza essere individuato. Cecchino stesso passò un'intera settimana appostato fuori dal castello, contando soldati e aspettato la buona occasione per assassinare una strega con una balestra.

La nostra investigazione indicava la presenza di tre streghe che possedevano i corpi di belle giovanotte paesane; una aveva appena sposato il barone, le altre due erano le sue vecchie amiche e damigelle. Il barone era completamente sotto il loro controllo magico e, sebbene la situazione fosse ovviamente controversa dal punto di vista sociale, le streghe erano comunque saldamente integrate a corte. La cosa peggiore era che il barone stava organizzando di mettersi in viaggio per presentare la sua nuova moglie al re di Moldavia la settimana successiva! La posta in gioco si era alzata bruscamente, era piuttosto ovvio quale fosse l'obiettivo di lungo termine qua.

La situazione sembrava irrisolvibile e il gruppo stava già pianificando la fuga da quella che sembrava una situazione strategicamente impossibile, quando mi venne l'idea che avrebbe salvato la Moldavia: avremmo dovuto incastrare le streghe per stregoneria. Loro erano chiaramente molto attente nel rispettare i propri ruoli al castello, ma in quanto comuni paesane avrebbe dovuto essere possibile fargli rivoltare contro la corte con qualche pantomima occulta ben orchestrata.

L'idea si trasformò in un vero e proprio piano per un colpo, col in gruppo che eseguiva operazioni parallele e sincronizzate nel tentativo di rubare il cadavere della baronessa deceduta e portarlo di nascosto nei quartieri di quella nuova, e allo stesso tempo cercava di hackerare socialmente la corte per preparare le circostanze della scioccante rivelazione.

L'esecuzione dello strategemma fu puntuale, anche se la scoperta che la nuova baronessa-strega tenesse un \textit{leopardo domestico} nelle proprie stanze si rivelò una piccola complicazione; perdemmo Pjotr il Ladro lì. A parte quello, però, andò tutto sorprendentemente liscio, sebbene al prezzo della vita di alcune guardie. Cecchino e un altro complice, Durriger, fecero pure una visitina al tesoro del barone mentre lasciavano il castello.

Le streghe erano un pubblico freddo, come ci si aspetta da degli scambia-corpi immortali, ma la nostra elaborata messinscena fu sufficiente a far perdere loro la calma e farle tirar fuori gli artigli; a quel punto avevamo essenzialmente tutto quello che volevamo: l'intero castello come testimone di magia malvagia. Non avevamo davvero dei mezzi che ne prevenissero la fuga, ma almeno avevamo salvato la baronia dalle grinfie di Satana! Nessuno scoprì mai che avevamo incastrato le streghe per stregoneria, ammazzato qualche guardia del castello per farcela e derubato il barone già che eravamo lì. Un colpo perfetto!

Il prossimo grosso sviluppo nella "costruzione del carattere" di Cecchino avvenne più avanti, quando ebbe la fortuna di trovare un serpente velenoso e oracolare (lascia che ti morda, se sopravvivi al tiro salvezza sul veleno divinerai la risposta a una domanda), che il gruppo