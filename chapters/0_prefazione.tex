\chapter{Prefazione}

\section{Nota del traduttore}

\textsc{Muster} fa ampio riferimento alla tradizione del "wargaming", cioè la simulazione di operazioni militari sotto forma di attività ludica. Siccome "simulazione di operazioni militari sotto forma di attività ludica" risulta alquanto lungo e macchinoso da ripetere continuamente e "giochi di guerra" suona infantilizzante e anche un po' artefatto, ho deciso di presentare questa breve nota all'inizio del testo, per chiarire le mie scelte, e poi mantenere il termine "wargaming" (che comunque è stato adottato da diversi contesti italiani) all'interno del testo.

Un altro termine che ho scelto di non tradurre, perché ormai insediatosi nel gergo ludico italiano come se ci fosse nato, è "ruling", la decisione presan dall'arbitro di gioco su un caso specifico non coperto dal regolamento (o in cui la soluzione suggerita dal regolamento va in una direzione che non ha senso per la situazione di gioco) e che, in un certo senso, "fa giurisprudenza" per la campagna in corso.

Come molti testi che parlano di GDR e \textit{wargames}, \textsc{Muster} utilizza il sistema imperiale. Siccome la conversione è triviale e non viene svolta in tutti gli adattamenti, soprattutto in contesti Old School, ho deciso non effettuarla all'interno del testo. Se volete fare conti grossolani, considerate 1 piede come 30 cm e 10 piedi come 3 m.

\subsection{Riferimenti e citazioni}

Come ogni testo che si appoggi così tanto sulla vecchia scuola di \dnd{}, \textsc{Muster} fa ampio riferimento alle edizioni passate del gioco di ruolo più famoso del mondo. Si tratta di una storia complessa in lingua originale e ancora più complessa se teniamo conto della sua disponibilità in Italia. Una trattazione dettagliata di queste due storie parallele va ben oltre non solo le mie capacità, ma anche lo scopo di \textsc{Munster} stesso. Tuttavia, per venire incontro al lettore che non intenda vedere l'alba tra Wikipedia e e-commerce vari, ho deciso di aggiungere qualche informazione utile, quantomeno, a favorire il reperimento dei testi citati.

Per le avventure, citate in gran numero e spesso mai pubblicate in italiano, ho raccolto quel che potevo nell'\hyperref[appendix:ita:avventure]{Appendice A}.

Per quanto riguarda i regolamenti, il testo originale presenta già una breve storia editoriale di \dnd{}. Ho fatto del mio meglio per integrarla in nota con i riferimenti alle edizioni italiane.