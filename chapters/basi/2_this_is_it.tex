\usection{Questo è quanto}

Com'è la via del \textit{wargaming}? Teemu ha voluto trascrivere una delle sue storie di guerra, una sequenza di partite di cui i giocatori hanno ancora ricordi ad anni di distanza. Da quello che ne so, questo è successo nei tardi '10 in \textsc{The Screams for Jedder's Hole}, un'avventura di Dyson Logos. Sentiamo:

\textbf{Le nostre prospettive di successo} erano, francamente, non buone.

Sono passati alcuni anni da queste sessioni, ma l'idea generale è che il nostro gruppo di buoni a nulla, avventurieri e mercenari era stato reclutato per rimediare alle sofferenze di un villaggio remoto: un gigantesco buco si era aperto nel pavimento della loro chiesa, rivelando catacombe dimenticate e tunnel da qualche fangosa profondità della storia; dal buco saliva uno strano e orribilmente affascinante lamento. I locali non erano entusiasti da questo stato di cose, così avevano arruolato qualsiasi canaglia potessero trovare per farsi aiutare a porvi fine.

Le esplorazioni iniziali nelle catacombe avevano rivelato che il posto era parecchio letale. Orrori non morti e ghoul cannibali avevano reclamato un prezzo piuttosto alto alle prime spedizioni; era chiaro che in questo luogo eravamo in svantaggio. Questa non era una situazione nuova; la campagna non aveva alcuna pretesa di avere avventure appropriate al livello, ed era sempre compito nostro, come giocatori, di valutare rischi e ricompense, se la sfida davanti a noi fosse troppo o fosse fattibile e quindi ritirarci o procedere. In questo caso, decidemmo di procedere.

Nonostante le perdite aumentassero costantemente, gli esploratori raggiunsero finalmente l'origine dell'orribile gemito: una massiccia creatura, delle dimensioni di una piccola casa, risiedeva nelle camere più profonde, cantando il suo orrido urlo-canzone. Questa era già una brutta notizia, ma il luogo in cui la bestia si trovava rendeva le cose molto, molto peggiori: stava in agguato in una stanza alla fine di un tunnel diritto, che non forniva alcun tipo di copertura al gruppo.

La situazione si rivelò letale quando provammo ad avvicinarci la prima volta; il mostro aveva lunghi tentacoli spaventosamente forti e ci attaccava senza pietà quando ci avvicinavamo, acchiappando qualche sfortunato in prima linea e mandando tutti gli altri in una rotta disordinata. Anche un breve primo contatto ci rivelò che non c'era speranza di avvicinarci senza venir spappolati contro un muro e trascinati nella bocca dell'orrore.

I sopravvissuti fuggirono in superficie e ora avevamo un obiettivo concreto, ma anche una sfida tattica piuttosto spaventosa. Se anche fossimo riusciti ad avvicinarci, come avremmo ucciso la bestia senza nome? Era il nostro gruppo di avventurieri mortali e piuttosto mondani contro un qualche tipo di orrido abominio che poteva ucciderci a suo piacimento, sul fondo di un dungeon molto pericoloso e all'interno di una stanza che, tatticamente, era una trappola mortale. Non potevamo avvicinarci, e se anche avessimo potuto, ci avrebbe probabilmente uccisi tutti.

Come ho detto, le nostre possibilità di successo non parevano affatto buone. Tuttavia, avevamo deciso di tentare e non eravamo ancora pronti ad arrenderci. Eravamo liberi di fare come volevamo, in questa partita, ma significava anche che era nostra responsabilità inventarci una risposta al problema: il nostro GM non ci avrebbe certamente favoriti in alcun modo.

Alla fine, dopo molte deliberazioni, il gruppo produsse un piano che era forse un po' ridicolo, ma allo stesso tempo sembrava molto più fattibile e possibile di qualsiasi altra balzana idea che si sarebbe ridotta a "carichiamo davvero veloce e speriamo che non possa ucciderci tutti". Il piano: avremmo costruito un muro mobile e l'avremmo usato per avanzare lungo il tunnel. Il mostro non avrebbe potuto attaccarci coi tentacoli da dietro il muro e, quando ci fossimo avvicinati abbastanza, il gruppo avrebbe potuto usare il muro come uno scudo e lanciargli addosso olio in fiamme mentre era al riparo. Fuoco e mura, i due grandi livellatori dell'umanità, e noi li avremmo usati entrambi a nostro vantaggio.

Il piano richiese uno sforzo piuttosto serio. Impiegammo interi giorni di gioco organizzando quell'incubo logistico, radunando nuovi avventurieri per lavorare al progetto, portando il muro in posizione (dopo averlo letteralmente segato via da un fienile), preparando argani così da poterlo calare nelle catacombe e tutto quello che potete immaginare. Ma alla fine, avevamo la nostra copertura tecnicamente mobile pronta a partire. Il piano sembrava leggermente ridicolo e mi ricordo che ne ridevamo, ma contemporaneamente era il meglio che potessimo fare e non sembrava totalmente impossibile da portare a compimento. Avremmo scoperto la verità, in un modo o nell'altro, una volta arrivati lì.

Mettere in moto il nostro muro e farlo arrivare fin giù nel dungeon fu un mezzo miracolo - per non dire che ci volle una quantità orribile di lavoro e dolore. Avevamo bisogo di molte persone solo per spostarlo e i mostri nei tunnel fiutarono la nostra vulnerabilità durante questo compito. Perdemmo parte del nostro gruppo per colpa dei ghoul e un altro sfortunato finì coperto di olio in fiamme nel caos.

Si trattava di cose perfettamente normali: sapevamo fin da subito che questo piano avrebbe probabilmente prodotto qualche vittima e comunque è raro che si veda un commando eseguire questo tipo di operazioni disperate senza perdite. Ma la cosa importante è che riuscimmo a spingere il muro attraverso tutti i tunnel e finalmente in posizione, vicino al tunnel che portava alla tana del mostro gemente. L'olio era pronto. I più forti del gruppo presero posizioone al muro. Avevamo fatto tutto quel lavoro solo per arrivare in posizione. Eravamo pronti, eravamo determinati. Ora di muoversi.

La cosa buffa dei piani di battaglia è che non sai come andranno finchè non arriva il momento fatale in cui sono messi alla prova. Si tratta di uno dei principali motivi per cui giochiamo a questo gioco: per scoprire il modo giusto di fare le cose e vedere se il piano regge o meno davanti a una vera opposizione. Il primo contatto col nemico ti mostrerà la validità dei tuoi progetti, nel bene e nel male. E in questo caso, scoprimmo presto che, in quel letterale primo contatto, quando i tentacoli del mostro colpirono il nostro muro, nessuno di noi era effettivamente in grado di tenerlo su contro la forza del mostro!

Un forte colpo e l'intero maledetto muro ci cadde in testa, accompagnato da panico e urla! Alcuni finirono intrappolati sotto al muro, altri riuscirono a scartare di lato. Tutto si trasformò in assoluto caos e orrore quando perdemmo la nostra copertura. I tentacoli frustavano il tunnel e colpivano uomini. Alcuni fuggirono immediatamente in preda al panico; altri presero rapidamente l'olio in fiamme da lanciare al mostro, in un ultimo, disperato tentativo di salvare il piano. Nel panico, la maggior parte dei contenitori scivolò da mani tremanti e si infranse nel caos e improvvisamente era tutto in fiamme - soprattuto quei poveri bastardi intrappolati sotto alle rovine.

I pochi che poterono, scapparono urlando nelle tenebre, inseguiti dall'eco dei lamenti dei loro compagni che bruciavano viv sotto il muro che avevano così faticosamente trasportato. Alcuni di noi arrivarono vivi fino in superficie e, a quel punto, era evidente che non eravamo adatti all'impresa. Dicemmo ai locali di sigillare il buco col cemento e ce ne andammo. Un uomo è libero di scegliere le sue battaglie e per alcune battaglie non ne vale affatto la pena.

In questo gioco, a volte tutti i tuoi sforzi e la pianificazione non produrranno altro che miseria. Questo va bene. Potresti scoprire che il piano aveva un problema o che la sfida che avevi scelto di affrontare era semplicemente troppo e i tuoi sforzi erano sempre stati destinati a fallire - o magari che, semplicemente, una sfortuna orribile e qualche dettaglio secondario si sono rivelati la tua fine. Ma, in ogni caso, questa è la prospettiva che accettiamo quando scegliamo le nostre battaglie e affrontiamo queste sfide: vittoria o fallimento reali.