\usection{Com'è}

Per il grognard, è naturale come l'aria. Ma qui scriverò per il nuovo arrivato. Una vista da fuori di quello in cui ti stai infilando quando ti unisci a una campagna di \dnd secondo la via del \textit{wargaming}.

Se sai già come si gioca ai giochi di ruolo, allora potrebbe convenirti prestare attenzione a ciò che viene descritto diversamente qui. Queste differenze sono genuinamente importanti.

\usubsection{Il gioco è un progetto creativo}

Spesso una nuova campagna è iniziata da un motore primo creativo, un certo \textbf{Game Master} volenteroso e appassionato che inizia una partita di \dnd. Lavorerà per preparare lo scopo e gli obiettivi del futuro gioco: cose come il setting letterario e lo specifico scheletro di regole da usare. Inoltre, si occuperà del materia d'avventura o delle sfide che presenterà più avanti al gruppo di gioco.

Oppure, potrebbe esserci un \textbf{gruppo, club o circolo} preesistente di giocatori appassionati. Organizzati già da tempo, concordano un piano per mettere in piedi una nuova campagna. Alla fine, emergerà un GM che si farà carico della responsabilità per la maggior parte del lavoro creativo, ma ci saranno idealmente anche altri: anfitrioni, segretari e capi del gruppo.

E i giocatori si uniranno al gruppo, come quando si fonda un nuovo gruppo o si dà inizio alla produzione di un film, con l'obiettivo di migliorare rispetto ai tentativi passati e fare meglio dell'ultima volta.

\usubsection{Presta attenzione al viaggio}

Sii rispettoso quando ti unisci a una nuova partita. La cultura del GDR dà per scontate molto cose, ma nella realtà non c'è un solo modo di giocare e dovresti guardare ai partecipanti più rodati per imparare come le cose si fanno in questo gruppo, questa volta.

Il GM è sempre, prima e primariamente, un maestro di cerimonie, col compito di facilitare il gioco: illustrare l'agenda creativa, insegnare le regole e consigliare come procedere di volta in volta. Se, in un qualsiasi momento, non capisci qualcosa o hai qualsiasi altra preoccupazione, chiedi direttamente al GM. È genuinamente responsabile per il gioco, tanto quanto ogni consigliere di un club è responsabile per le attività appropriate.

Presta attenzione al gioco e al tuo comportamento al tavolo. Il rispetto è la chiave per essere graditi ed essere graditi è il primo passo verso il cameratismo.

Una campagna ben gestita è aperta a visitatori e nuovi partecipanti. La partecipazione regolare è grandemente appressata, ma semplicemente vedere facce nuove è sempre un piacere, quindi non esitare a stabilire i tuoi termini di partecipazione, anche se sono semplicemente "occasionalmente" o "solo questa volta".

\usubsection{Generare un personaggio}

Il tuo primo compito in gioco è generare un personaggio, la pedina che userai per partecipare al gioco. Un avventuriero fittizio che affronterà pericoli per la gloria.

Nello stile vecchia scuola della via del \textit{wargaming}, si parla di \textit{generare}, non creare, che è un'importante distinzione creativa rispetto ai giochi di ruolo più comuni: nella maggior parte dei GDR odierni, il tuo personaggio è importante e sentito; un piccolo progetto creativo a cui si dedica il giocatore. Non è così nella via del \textit{wargaming}: il tuo personaggio vale poco ed è, alla fin fine, sacrificabile e accumula valore solo attaverso la sopravvivenza durante le partite.

Di conseguenza, la generazione di un personaggio dovrebbe essere un processo piuttosto semplice, che tradizionalmente consiste in poco più che tirare una linea di statistiche (una lidea di punteggi di abilità casuali creati con i dadi) e selezionare la classe del personaggio (una specializzazione tattica). Tutto il resto che potrebbe emergere in una specifica campagna sono dettagli, un abbellimento non necessario di quella linea di caratteristiche casuali e della scelta della classe.

I giocatori, a questo punto, possono intrattenersi dando nomi e assegnando personalità ai loro personaggi, magari facendosi qualche idea su chi siano e da dove vengano nel mondo fittizio del gioco. Questo è concesso ma non andrebbe incoraggiato, perché questi personaggi esistono per morire. Un giocatore che dedica lavoro emotivo in un personaggio presto e con impegno sta corteggiando la tristezza. Un veterano del gioco estenderà questi affetti con attenzione e nel tempo, una volta che il personaggio ha dimostrato una certa tendenza a sopravvivere.

\usubsection{Viene introdotta un'avventura}

Una volta che i giocatori hanno i loro personaggi pronti, il GM descriverà lo scenario del giorno. Potresti unirti a una campagna avanzata, con un posizionamento complesso che sopravvive da una sessione all'altra. Sessioni simili iniziano tipicamente con il GM che riassume lo stato del gioco e poi si spostano sui partecipanti regolari che organizzano un dibattito di pianificazione su come lo scenario in gioco dovrebbe svilupparsi.

Ma dando per scontato un gioco semplice, quello che il GM ti racconterà sarà invece la premessa dell'avventura che ha preparato per voi. Sarà un "dungeon", una località misteriosa lontana dalla civiltà. Ci sarà una ragione per cui un gruppo di avventurieri andrà a cercare quel dungeon, spesso collegata alla ricerca di ricchezze nascoste o a combattere un grande male.

La prima parte della partita consiste nel pianificare e preparare il vostro gruppo di avventurieri per una spedizione nel dungeon del GM. I giocatori hanno l'un l'altro e i loro personaggi, un po' di soldi per comprare le scorte e il proprio ingegno per muovere in qualsiasi modo venga loro in mente. Magari cercherete dicerie e conoscenze, in modo da avere un'idea di cosa vi aspetta nel dungeon; o forse vi affretterete verso il dungeon per vederlo coi vostri occhi. La scelta sta al gruppo.

Il GM vi aiuterà e supporterà nel fare i preparativi. Se il viaggio verso il dungeon è lungo e difficoltoso, sottolineerà certamente il bisogno di cibo e attrezzatura da campeggio. Vorrà che abbiate con voi le scorte basilari come torce e armi, altrimenti non ci sarà una gran sfida tra dungeon e gruppo. In ogni caso, il GM non è il vostro manager: quel ruolo verrà reclamato da un membro del gruppo, o non verrà reclamato affatto. Non aspettarti che il GM prenda decisioni o giochi al posto vostro.

\usubsection{Uomo contro Dungeon}

Una volta che il gruppo è pronto per andare all'avventura, il GM discute l'effettivo ingresso nello scenario. Questo è il momento in cui, essenzialmente, iniziate a giocare a un \textit{wargame}, qualcosa di simile agli scacchi ma a complessità infinita: i giocatori muovono contro l'edificio relativamente statico del dungeon e delle sue sfide preparate dal GM.

Ci possono anche essere mostri attivi che manovrano anche loro, con il GM che tiene segretamente traccia delle loro azioni. Per la maggior parte, la natura fondamentale del gioco è quella di un gruppo attivo di avventurieri che monta un assalto contro le difese statiche di una località sotterranea fantastica e misteriosa.

I punti di forza del dungeon in questa sfida sono molti: ha mostri possenti e innumerevoli servitori e i suoi mezzi magici sono incognite ignote, col potenziale di sorprendere gli intrusi in qualsiasi momento. Il dungeon è capriccioso, ma il GM gioca pulito: ha appunti e tracce di quello che il dungeon ha a disposizione e di come usa i propri mezzi. Il GM è lì pronto a sollevare progressivamente la nebbia di guerra, man mano che l'esplorazione avanza, non per decidere arbitrariamente quello che succede.

Anche il gruppo di avventurieri ha i suoi vantaggi: nello scenario più semplice, l'offensiva spetta a voi, scegliendo come e dove colpire. Potreste sorprendere gli abitanti del dungeon. E avete a disposizione il balsamo della ritirata, che è come il più grande degli imbrogli: manovrando saggiamente, è possibile costringere il dungeon a scoprire le proprie carte per poi fuggire con quella conoscenza.

\usubsection{L'esplorazione ha una struttura}

La sfida formidabile del dungeon è strutturata secondo le effettiva regole del gioco: il gruppo si muove attraverso il dungeon in Turni di esplorazione (che rappresentano segmenti da 10 minti di tempo in gioco), consumando luce, acqua e altre scorte, mentre mappano il luogo e trovano tracce dei pericoli che li attendono. Vengono scoperte porte, che nascondo stanze che contengono la promessa di mostri, tesori e molto altro.

Il gruppo di avventurieri si organizza internamente per questa fase di gioco, con uno dei giocatori che tipicamente si occupa della mappa, un altro che tiene traccia di tempo e scorte e uno che agisce come leader e capitano dell'azione, raccogliendo le opinioni e annunciando le decisioni del gruppo al GM.

Vi troverete spesso ad essere più di due o tre giocatori, il che vuol dire che, tipicamente, non tutti staranno costantemente facendo cose durante la partita. Avrete il lusso di stare un po' in disparte e seguire l'azione, osservando e pensando. Quando vi viene un'idea degna di nota, è il momento di fermare l'azione e farlo sapere agli altri giocatori. Siete un team, dopo tutto, che sfida il dungeon assieme.

\begin{figure}[p!]
    \resizebox{\linewidth}{!}{
        \begin{tikzpicture}
            \justifying
            \node[text width=\linewidth] (T) at (-1, 0.25) {
                \textsc{\LARGE{Struttura basilare di una sessione}}\\
                \textit{Le campagne di \dnd{} procedono come una serie regolare di sedute chiamate \textbf{sessioni}. La durata tipica di una sessione è di circa 4 ore, sebbene sessioni dalla durata doppia che occupano un'intera giornata non siano così rare. Una campagna a ritmo pieno gioca tutte le settimane, ma la sessione basilare è adatta anche a one-shot e sessioni improvvisate.}
            };
            \node[phase] (F1) at (-4,-2.2) {Il GM introduce lo scenario};
            \node[phase] (F2) at (2,-2.2) {I giocatori preparano il gruppo};
            \node[comment] (C2) at (-4,-4) {Discuti l'ambientazione e imposta gli obiettivi dello scenario. Illustra le regole e rispondi alle domande mentre i giocatori si preparano};
            \node[comment] (C3) at (2,-4) {Crea i personaggi, assolda mercenari, equipaggia il gruppo, revisiona le informazioni, fai piani\dots};
            \node[phase] (F3) at (-3.5,-7) {Esplorazione del dungeon};
            \node (E3) at (-1.5, -6.5) {};
            \node (E4) at (-0.1, -7) {};
            \node (E5) at (-1.5, -7.5) {};

            \node[comment, text width=3cm] (C4) at (-4,-8.8) {La parte principale della partita: il gruppo naviga il dungeon, mappando ed esplorando.};
            \node[comment, text width=4cm] (C5) at (1.3,-8.8) {Mentre il gruppo scopre mostri, trappole, tesori e stranezze nel dungeon, il gioco si concentra su questi eventi.};

            \draw[->, ultra thick]
                  (E3) to[arc through={clockwise,(E4)}] (E5);
            \node[phase] (F4) at (1,-7) {Incontri \& scoperte};

            \node[phase] (F5) at (-4.4,-11) {Ritirata \& calcolo punti};
            \node[comment, text width=4.5cm] (C6) at (-4.4,-13.5) {Gli avventurieri si ritirano dal dungeon per via dei rischi, perché hanno ottenuto i loro obiettivi o perché la sessione termina. L'arbitro dà un punteggio ai risultati del gruppo.};
            \node[phase] (F6) at (2.2,-11) {Muoiono tutti};
            \node[comment] (C7) at (2.2,-12.8) {Il mondo sotterraneo dei dungeon è pericoloso, errori e sfortuna possono far terminare lo scenario con morti ignominose.};
            
            \draw [->,line width=2pt] (-2,-2) -- (0,-2);
            \draw [<-,line width=2pt] (-2,-2.3) -- (0,-2.3);
            \draw [->,line width=2pt] (-1, -2.4) 
                    -- (-1, -5.5) 
                    -- (-3.5, -5.5)
                    -- (F3.north);

            \draw[->,line width=2pt] (-2.15, -7.5)
                 |- (F5.east);
            \draw[->, line width=2pt] (-1.8, -7.5) |- (F6.west);
            
        \end{tikzpicture}
    }
\end{figure}

\usubsection{Essere un buon giocatore}

Il giocatore ideale presta attenzione alla situazione di gioco mentre si sviluppa; i giochi di ruolo sono giocati per lo più dialogando e prendendo appunti.

Ciononostante, fare domande è ancora più importante che prestare attenzione. Chiedi che ti si ricordino dettagli che ti sono sfuggiti. Chiedi ulteriori dettagli. Il GM ci ha detto di che materiale è fatta quella porta? Il mio personaggio riconosce questi escrementi animali?

A volte il gruppo agisce unito e i giocatori discutono quale sia il miglior corso d'azione. A volte non c'è tempo di discutere e un giocatore confidente semplicemente dichiara un corso d'azione che viene poi seguito o meno degli altri. E a volte in singoli personaggi si dividono il lavoro.

Quando è il tuo momento di agire, pensa più in termini di manovre che di giorecitare. Molti giochi di ruolo si concentrano sulla giorecitazione: sappiamo tutti cosa farà il tuo personaggio, l'arte sta nell'esprimerlo in modo vivace. \dnd non è così: la cosa importante è che tu sia conscio della situazone generale e che l'azione che scegli di compiere vi sia coerente. A nessuno importa che tu descriva l'azione del tuo personaggio come un abile narratore, quello che è importante è che sia la cosa giusta da fare.

E tieni gli occhi aperti per delle opportunità Questo è un gioco molto diverso dai giochi da tavolo, nel senso che non ci sono veri turni tra i giocatori. Puoi passare un'intera sessione, lunga ore, seduto senza fare nulla di più che ascoltare gli altri mentre gestiscono il gioco. In effetti, questo va perfettamente bene. Ma non stare seduto con l'idea che, se resti seduto abbastanza a lungo, qualcuno si girerà verso di te e ti chiederà di giocare il tuo turno. Quel turno non arriverà mai. Ci si aspetta che tu intervenga quando hai qualcosa da dire. Si applicano le regole dei gruppi di lavoro, non dei giochi da tavolo.

Avventure che altrimenti non andrebbero da nessuna parte vengono spesso salvate da una singola buona idea che viene da un giocatore silenzioso. Questo può sembrare controintuitivo, ma in realtà è molto sensato: i giocatori che sono impegnati a gestire i movimenti del gruppo minuto per minuto non hanno il tempo di pensare e prestare attenzione alla situazione generale.

\begin{figure}[p!]
    \resizebox{\linewidth}{!}{
        \begin{tikzpicture}
            \justifying
            \node[text width=\linewidth] (T) at (-1, 0.25) {
                \textsc{\LARGE{Procedura di esplorazione del dungeon}}\\
                \textit{Questo è il cuore originario di \dnd{}, la tecnologia di} wargaming \textit{ usata per risolvere le escursioni nei dungeon. Rimane giocabile ed eccitante anche senza l'indoratura di classi personaggio uniche, o punti esperienza, o regole di combattimento tattico o persino trattative in dialogo libero. Una danza di geometria mappale, tempo e azzardo.}
            };
            
            % Movimento
            \node (A1_1) at (-2, -2.1) {};
            \node (A1_2) at (-3, -2.5) {};
            \node (A1_3) at (-2, -2.8) {};
            \draw[->, ultra thick]
                 (A1_1) to[arc through={counterclockwise,(A1_2)}] (A1_3);
            \node[action] (movimento) at (-4,-2.5) {Movimento};
            \node[comment, text width=3.5cm] (c_movimento) at (-4.5, -4.5) {Scegliete direzione e velocità; la velocità di esplorazione è più sicura e permette di tenere una mappa, ma copre meno terreno};

            %Riposo
            \node (A2_1) at (-1, -6.2) {};
            \node (A2_2) at (-2.2, -6.3) {};
            \node (A2_3) at (-1.1, -6.6) {};
            \draw[->, ultra thick]
                 (A2_1) to[arc through={counterclockwise,(A2_2)}] (A2_3);
            \node[action] (riposo) at (-2.7,-6.5) {Riposo};
            \node[comment, text width=4cm] (c_riposo) at (-4, -7.9) {Ogni 5 Turni e dopo gli scontri. Tenete traccia di torce e acqua.};

            %Ricerca
            \node (A3_1) at (1.5, -4.2) {};
            \node (A3_2) at (2.5, -4.5) {};
            \node (A3_3) at (1, -4.8) {};
            \node[comment, text width=3cm, align=right] (c_ricerca) at (3.5, -6.8) {Perquisici attentamente una stanza o un corridoio per segni, indizi, trappole o tesori nascosti.};
            \draw[->, ultra thick]
            (A3_3) to[arc through={counterclockwise,(A3_2)}] (A3_1);
            \node[action] (ricerca) at (2.8,-5) {Ricerca};

            %Incontri casuali
            \node [phase] (incontri) at (-4, -9.5) {Incontri casuali};
            \node[comment, align=left, text width=4cm] (c_incontri) at (-4, -11.5) {L'arbitro controlla ogni turno o quasi se gli abitanti del dungeon hanno scoperto il gruppo. Questa è raramente una buona notizia.};

            %Scoperta
            \node[main] (scoperta) at (2.4,-10) {\Large{Scoperta}};

            % Turno di esplorazione
            \node [comment, align=left] (c_main) at (3.1,-2.3) {Circa 10 minuti in gioco. Il gruppo decide la sua prossima mossa e il leader informa il GM.};
            \node[main] (main) at (-0.7,-4) {\Large{Turno di esplorazione}};

            %Sorpresa
            \node (A4_1) at (0.8, -10.5) {};
            \node (A4_2) at (-0.5, -11) {};
            \node (A4_3) at (0.8, -11.5) {};
            \node[comment, text width=3.5cm, align=right] (c_sorpresa) at (-0.5, -13) {Gli avventurieri possono incontrare una gran varietà di cose ed esseri nel dungeon.};
            \node[comment, align=center] (c_sorpresa_1) at (-0.5, -14.9) {Il controllo tattico dei termini di ingaggio è tanto importante quanto la scelta dell'effettivo corso d'azione\dots};
             \draw[->, ultra thick]
                (A4_1) to[arc through={counterclockwise,(A4_2)}] (A4_3);
            \node[action] (sorpresa) at (-0.5,-11) {Sorpresa};

            \draw [->, line width=2pt] (3.5,-11) -| (5,-16);

            \node[action,text width=3cm] (parlamentare) at (5,-12) {Parlamentare};
            \node[action,text width=3cm] (lottare) at (5,-13.5) {Lottare};
            \node[action,text width=3cm] (fuggire) at (5,-15) {Fuggire};


            \draw [->,line width=2pt] (-0.8, -6.3) |- (-2, -9.2);
            \draw [->,line width=2pt] (-0.6, -6.3) |- (0.6, -9.2);
            \draw [->,line width=2pt] (-2, -9.6) |- (0.6, -9.6);
        \end{tikzpicture}
    }
\end{figure}

\usubsection{L'avventura finisce}

Prima o poi, la vostra avventura nel dungeon terminerà in uno di due modi: in un disastro, se qualche pericolo del dungeon reclama le vite degli avventurieri, o in ritirata, se gli avventurieri concludono che è il momento di andarsene. Nel caso migliore, ve ne andate avendo portato a termine la vostra missione, carichi di tesoro o con la principessa rapita al seguito. Spesso ve ne andrete a malapena vivi, cosa che potrebbe venir considerata un pareggio inconclusivo tra dungeon e avventurieri.

Questo è un gioco e nei giochi ci sono i punteggi; alla vine di ogni avventura il gruppo conta l'ammontare del proprio successo sotto forma di "punti esperienza". Questi punti si ottengono per aver portato a termine gli obiettivi dello scenario; spesso si tratta di portare via quanto più tesoro possibile. Un calice d'oro, 150 PX, cha-ching.

I punti ottenuti da un'avventura di successo sono divisi tra il gruppo di avventurieri, o quantomeno i suoi sopravvissuti, di solito più o meno alla pari. Ottenere qualche centinaio di punti per un'escursione di una notte è piuttosto buono, ottenerne qualche migliaio è eccellente.

L'obiettivo del gioco: sopravvivere e prosperare. I personaggi che hanno successo in diverse avventure diventano più potenti e famosi nel mondo di gioco, cosa che, assieme alla crescita delle capacità del giocatore, rende le future escursioni progressivamente più facili. E quando finalmente la sete di un avventuriero per nuove ricchezze e conquiste è placata, allora il gioco è finalmente vinto. A dire il vero, perdiamo più spesso di quanto non vinciamo; è difficile sopravvivere al dungeon una spedizione dopo l'altra. Prima o poi, anche i migliori giocatori fanno errori.

\begin{figure}[p!]
    \resizebox{\linewidth}{!}{
        \begin{tikzpicture}
            \node[text width=\linewidth] (T) at (-1, 0.25) {
                    \textsc{\LARGE{Successi di inizio gioco}}\\
                    \textit{Il gioco è vasto e come prima cosa dovresti concentrarti sull'osservare gli altri per imparare a giocare. Dopodiché, la cosa migliore è giocare in maniera ambiziosa e orientata agli obiettivi; cogli l'attimo!}
                };
            \node[draw, ultra thick, rectangle,font=\Large\scshape] (A) at (-1, -1.4) {Spedizione di Successo};
            \node[draw, ultra thick, rectangle, text width=\linewidth] at (-1, -3.3) {\textit{Incassa 100 MO}\\ I personaggi iniziali sono piuttosto poveri per gli standard degli avventurieri. Una singola spedizione di successo può risolvere questo problema, aprendo nuove possibilità: assoldare gregari, acquistare equipaggiamento e dedicare del tempo a preparare la prossima spedizione.};

            \node[draw, ultra thick, rectangle,font=\Large\scshape] (A) at (-1, -5.4) {Raggiungi il 2° Livello};
            \node[draw, ultra thick, rectangle, text width=\linewidth] at (-1, -7.55) {\textit{Porta un personaggio a $\sim$2000 PX.}\\ Come ottenere una cintura nelle arti marziali, è un divisorio formale tra casual e regolari. Nella mia esperienza, ci voglio 10-20 sessioni di tentativi, errori e abilità, salvo gli occasionali colpi di fortuna. Il gioco che pratichi per arrivare al secondo livello è tutto il \dnd{} che conta, il fondamento del gioco di fascia bassa. Tutto il resto si costruisce su queste lezioni.};
            \node[draw, ultra thick, rectangle,font=\Large\scshape] (A) at (-1, -9.9) {Fai tutto};
            \node[draw, ultra thick, rectangle, text width=\linewidth] at (-1, -12.45) {
                \begin{tikzpicture}
                    \node[text width=0.4\linewidth, font=\itshape] at (-1, 0) {Avvia una raccolta di personaggi.\\Pensiona un personaggio.\\Ottieni un seguace.\\Diventa famoso.\\Cavalca un drago.\\Dichiara il tuo Nome.};
                    \node[text width=0.4\linewidth] at (4.5, -0.3) {E, ovviamente, tieni conto dell'obiettivo generale della campagna. Spesso è solamente di portare un avventuriero tanto lontano quanto può arrivare, ma non è il caso di ogni campagna.};
                \end{tikzpicture}
            };
        \end{tikzpicture}
    }
\end{figure}