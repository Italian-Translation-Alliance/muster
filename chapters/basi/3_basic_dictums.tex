\usection{Principi di base}

\begin{figure}[p!]
    \resizebox{\linewidth}{!}{
        \begin{tikzpicture}
            \node[text width=\linewidth, align=center] (T) at (-1, 0.25) {
                    \textsc{\LARGE{Il Muro della Saggezza della Via del Wargaming}}\\
                    \textit{Alcuni detti da attaccare alle pareti del tuo dojo.}
                };
            \node[draw, ultra thick, rectangle,font=\Large\scshape] (A) at (-1, -1.4) {Tre Pietre Angolari};
            \node[draw, ultra thick, rectangle, text width=\linewidth] at (-1, -3.3) {
                \begin{tikzpicture}
                    \node[text width=0.3\linewidth,font=\small] at (-1, 0.25) {
                        \textbf{Arbitro neutrale}\\
                        \textit{
                            Non si bara\\
                            Equilibrio dinamico\\
                            Procedure igieniche\\
                            Partecipazione autentica
                        }
                    };

                    \node[text width=0.3\linewidth,font=\small] at (2.3, 0.25) {
                        \textbf{Gioco simulativo}\\
                        \textit{
                            Ruling, non regole\\
                            Le regole sono residuati\\
                            Insegna la realtà, usa la realtà
                        }
                    };

                    \node[text width=0.3\linewidth,font=\small] at (5.6, 0) {
                        \textbf{Risultati reali}\\
                        \textit{
                            I giocatori vincono e perdono\\
                            I PE sono veri punti\\
                            Imparare la sportività\\
                            Imparare cose
                        }
                    };
                \end{tikzpicture}
            };
            \node[rectangle, text width=\linewidth] at (-1, -5.7) {
                \textit{La definizone dogmatica della via del} wargaming\textit{, in un certo senso. I tre principi chiave sembrano applicarsi all'intera tradizione del} free krieksspiel.
            };

            \node[draw, ultra thick, rectangle,font=\Large\scshape, text width=0.35\linewidth, align=center] at (-4, -7.2) {Gestione del Gruppo};
            \node[draw, ultra thick, rectangle, text width=0.4\linewidth, font=\itshape] at (-4, -10.2) {
                Creazione rapida dei personaggi\\
                Inizio al 1° livello\\
                Nessun background\\
                I personaggi muoiono/si ritirano\\
                Uso di raccolte di personaggi\\
                Uso di mercenari e seguaci
            };

            \node[draw, ultra thick, rectangle,font=\Large\scshape, text width=0.35\linewidth, align=center] at (2, -8) {Modello Simulativo};
            \node[draw, ultra thick, rectangle, text width=0.4\linewidth, font=\itshape] at (2, -11.4) {
                Basi realistiche\\
                Statistiche iniziali casuali\\
                I PE sono basati sugli obiettivi\\
                Il livello modella la statura eroica\\
                Progressione geometrica dei livelli\\
                Crescita lineare dei PF\\
                Crescita sub-lineare dei danni
            };
        \end{tikzpicture}
    }
\end{figure}

Quando si inizia una campagna di \dnd nello stile del \textit{wargaming}, questi sono i principi sottostanti a cui non penso nemmeno, da tanto sono fondamentali. Non deviare da questi principi a cuor leggero.

\paragraph{Non barare o modificare i tiri}. Le regole dovrebbero essere pubbliche e le decisioni giustificate. L'arbitro presenta le sfide correttamente e prende decisioni eque. Non decide chi meriti di vincere o di perdere, quello spetta al processo di gioco. I dadi entrano in gioco con il risultato che presentano e le aspettative dei giocatori si conformano a questa realtà. Magari non è bella, ma è sempre scrupolosamente giusta ed equa.\\
Questo significa che un sacco di personaggi moriranno. Un sacco di gruppi di avventurieri falliranno. Vincerai solo quando avrai dalla tua una combinazione di fortuna e abilità. Questo va bene, perché l'alternativa è un'opera teatrale in cui l'arbitro ti consegnerà la vittoria in base ai suoi capricci.

\paragraph{Inizia al primo livello}. L'intero obiettivo del gioco è esplorare il paesaggio cangiante dell'avventura mentre i personaggi crescono da persone ordinarie a grandi eroi. La moneta dell'eroismo è impoverita le terribili poste in gioco sono impedite da una posizione non guadagnata. Il gioco è più divertente quando lasci le decisioni in mano all'abilità dei giocatori e alle vagarie del fato.\\
Questo vuol dire anche far iniziare nuovi personaggi al primo livello, anche se significa che PG diversi avranno livelli diversi. Credimi, funziona.\\
Significa anche non trattare la generazione del personaggio come un esercizio creativo di grande profondità e personalizzazione. Metti in gioco quel personaggio, vedi se sopravvive e preoccupati di conoscere la sua faccia solo dopo.\\
Significa anche che la gran parte del gioco avviene ai livelli bassi. Il gioco non è una stabile ma faticosa salita ai livelli alti, il punto di partenza a cui torniamo dopo ogni fallimento è il primo livello e qualsiasi avanzamento dei personaggi oltre quello è una gloriosa eccezione. Un volo che deve, per necessità, tornare a terra.

\paragraph{Non c'è un gruppo}. La campagna esiste come piattaforma da cui il GM introduce nuovi scenari e magari per permettere agli avventurieri di muoversi tra uno scenario e l'altro. Non è una cronaca delle gesta e delle vite di uno specifico gruppo di avventurieri, un \textit{party}. In gruppi si formano in base alla necessità, spesso partendo da zero a ogni nuova sessione di gioco.\\
L'implicazione è che non ti serve un gruppo stabile di giocatori da una settimana all'altra. Non ti serve che i loro personaggi rimangano sempre gli stessi. Va bene che i personaggi muoiano e vengano rimpiazzati, va bene che un giocatore gestisca un repertorio di più personaggi. Se il paladino non vuole unirsi a un'avventura moralmente grigia, il giocatore può sempre fare un nuovo personaggio che lo farà.\\
Questo non vuol dire che non ci sarà continuità, solo che non è obbligatoriamente lo stato di default del gioco. Se i giocatori riescono a formare una compagnia solida che va assieme all'avventura da uno scenario all'altro, buon per loro.

\paragraph{L'ambientazione ha basi reali}. Vale a dire, che la campagna ha una comprensione consistente di come gli elementi ricorrenti sono simulati meccanicamente: i soldati ordinari hanno 1d6 punti ferita; scalare un albero richiede una prova di DES; i ghoul hanno 2 DV. Queste convenzioni non dipendono dai livelli degli avventurieri correnti o dalle opinioni del GM su quanto una situazione dovrebbe essere difficile per rendere un'avventura eccitante; sono quel che sono indipendentemente dai desideri umani.\\
Le basi sono così importanti perché formano il contesto per l'avanzamento eroico che è così prominente come argomento per \dnd. I personaggi e la loro relazione di potere con la realtà mondana dell'ambientazione cambiano, ma questo può essere vero solo se abiuri tutti i trucchetti meccanici abbracciati dai GDR più teatrali e lasci che l'ambientazione sia fissa a specifiche costanti meccaniche.

\paragraph{I PX sono scrupolosamente basati sugli obiettivi}. Questo significa che i punti esperienza, il meccanismo che traccia i punteggi del gioco, non si guadagnano in base al capriccio dell'arbitro o solo per essersi presentati, o per una buona interpretazione, o per aver partecipato a una scena eccitante. Solo il vero gioco fatto da veri giocatori, che porta al successo nell'avventura, merita punti.\\
Questo non vuol dire che i PX possono essere assegnati soltanto per aver saccheggiato tesori nei dungeon; il gioco ha il potenziale per obiettivi di scenario più variegati. Ma non vuol dire che gli eventi di gioco non sono ciò che assegna punti; si parla sempre e solo di ottenimento intenzionale di obiettivi.\\
Prendere sul serio l'assegnazione di PX è importante per rendere il gioco divertente per le stesse ragioni per cui lo sport prende i punteggi seriamente: è come vedi chi sta vincendo, e di quanto.

\paragraph{I giocatori vogliono fare punti}. Il gioco ha una chiara condizione di vittoria (sopravvivi e guadagna PX) e questo non dovrebbe essere oscurato da altere idee stilistiche come "interpretare bene il proprio personaggio". Il decoro sportivo è fondamentale, ovviamente, ma sapere cosa dà punti e come è uno strumento di orientamento così importante per il gioco che nulla dovrebbe ostacolarlo.\\
Questo significa che le regole per i punteggi sono pubbliche e note. E significa che i giocatori dovrebbero generalmente puntare a fare le mosse migliori possibili; se questo si beffa delle personalità già stabilite dei personaggi o della nebbia di guerra, usa delle regole per controllare queste evenienze, invece di far sentire in colpa i giocatori per aver cercato di giocare per fare punto.

\paragraph{Il gioco è autentico}. Il GM prepara obiettivi, situazioni, forze d'opposizione e terreni di manovra, non trame e scene e certamente non risultati. Gli eventi si risolvono seguendo le regole che li modellano, non pregiudizi o preconcetti su come le cose "dovrebbero" andare.\\
Questo significa che i giocatori scelgono le loro mosse basandosi sulla situazione che gli viene presentata, non i trope del genere "dungeon fantasy". Se la tua partita ricorda il \dnd storico è perché arrivate alle stesse conclusioni, non perché stai scimmiottando il passato.\\
Questo significa anche che il fallimento è tanto prezioso quanto la vittoria perché, qualsiasi cosa accada, è reale nel senso che discende, realmente, da quello che noi, i giocatori, stiamo facendo. Non siamo un pubblico, siamo gli agenti, anche quando l'azione va ridicolmente male.

\paragraph{Bilanciamento dinamico}. Questo è uno degli antidoti principali al bilanciamento del gioco: l'arbitro non deve preoccuparsi se l'avventura è troppo facile o troppo difficile perché sia significativa per il gioco, se il gruppo può negoziare e adattarsi alla sfida.\\
Per ottenere il bilanciamento dinamico bisogna scartare il concetto di incontri fissati e preparati dal GM. La sfida emerge dalla scelta dei giocatori di interagire e dedicarsi alle operazioni, non dai piani del GM. Il GM prepara un mostro, ma sono i giocatori che decidono se valga la pena di affrontarlo. La ritirata è sempre un'opzione.\\
E, come corollario: se qualcosa è facile, triviale, non giocabile per una mancanza di sfida, può essere sorvolato, risolvendolo con un processo sommario. Non sprecate il vostro tempo di gioco con l'irrilevante, andate a cercare la sfida.

\paragraph{I giocatori guidano il gioco}. Il GM-arbitro solitamente prepara il palco indicando come preparare il gruppo e introducendo lo scenario iniziale. Dopodiché, si tira indietro. "Questa è la situazione. Cosa farete dopo?" è il cuore del gioco, è tutto il gioco.\\
Questo significa che l'arbitro non è responsabile per dare un ritmo alla partita: quello spetta al leader del party. Non è responsabile della direzione: quello spetta alle decisioni del gruppo. Non è responsabile di divertimento e emozioni: se li volete, fate mosse divertenti ed eccitanti. L'arbitro è qui per risolvere eventi, non preoccuparsi di rendere divertente il gioco.\\
E significa anche che i giocatori hanno un potere reale negli eventi di gioco: decisioni reali, vittoria reale e sconfitta reale. L'arbitro è troppo neutrale, equo e orientato al processo per preoccuparsi dei \textit{risultati}.

\paragraph{Le dinamiche di risoluzione modellano la realtà}. Il processo di risoluzione del gioco non ha valori ed è privo di preconcetti, senza "privilegio dei PG" o "invenzioni del GM". Le regole e i \textit{ruling} sono basate su un tentativo onesto di modellare equamente fenomeni reali; la giocabilità è una considerazione per un livello completamente diverso del gioco, la modellazione dovrebbe puntare coraggiosamente a ciò che è vero piuttosto che a ciò che è conveniente.\\
Questo significa che preparare il gioco e migliorare al gioco è un viaggio didattivo verso un vero apprendimento. Non deve essere eccessivamente ambizioso, ma qualsiasi dettagli tu scelga di modellare, modella il mondo di gioco così com'è.\\
Molte parti importanti del gioco sono fantasiose, ma questo non ci libera affatto dall'ideale di voler modellare la realtà. Innanzitutto, modelliamo la fantasiosa realtà diegetica del gioco, non necessariamente la nostra; in secondo luogo, fantasia e magia hanno le loro regole e principi nella storia culturare di religione e mito, che possiamo modellare.

\paragraph{I \textit{ruling} sono sinceri}. La procedura di risoluzione è un \textit{ruling} fatto sul momento dall'arbitro; mentre il gioco si sviluppa, l'arbitro sceglie le regole specifiche e i concetti meccanici che verranno usati per risolvere gli eventi nella simulazione. L'arbitro serve onestamente, apertamente e, alla fine, per il piacere dell'intero gruppo.\\
Questo significa che qualsiasi regole userai nel tuo gioco, lo farai di tua spontanea volontà. E se il gruppo non è d'accorso su come le regole dovrebbero essere, si troverà un accordo da esseri umani.\\
E questo, alla fin fine, significa che l'arbitro dovrebbe, quando gli viene chiesto, spiegare ogni regola. La campagna contiene la sua propria autorità.

\paragraph{Le regole sono residuati}\translatornote{Ho fatto diversi tentativi per tradurre questo termine, ma non sono soddisfatto fino in fondo. Il testo originale usa ripetutamente il termine "cruft", un'espressione del gergo hacker (appare fin dalla prima edizione del dizionario del Tech Model Railroad Club del 1959), che indica artefatti (hardware o software) superati o di provenienza inspiegabile, che andrebbero rimossi ma rimangono.}\label{notes:cruft}. Vale a dire, il testo di gioco che avete in mano può al massimo essere la documentazione del processo di gioco di qualche campagna passata. Non può essere la base autoritative per il vostro gioco, perché elevarlo ad autorità priva il gruppo dell'autentico dovere da \textit{wargaming} di \textit{modellare davvero} lo scenario; modellare significa prendersene la responsabilità intellettuale e non puoi farlo se elevi le Regole Così Come Sono Scritte al di sopra della tua opinione onesta.\\
Questo non significa che leggere il testo delle regole non sia utile. Significa soltanto che trattare questi testi come qualsiasi cosa al di sopra di suggerimenti e ispirazione è un comportamento anti-igenico.