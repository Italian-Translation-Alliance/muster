\usection{\dnd{} e sciovinismo}

Abbiamo più o meno coperto quello che considero il materiale introduttivo a \dnd{}; la seconda metà di \textsc{Muster} avrà un approccio più avanzato, affrontando gli aspetti pratici che interessano arbitri e giocatori veterani.

Prima di procedere, però, dirò alcune parole su un argomento che sta perseguitando \dnd{} sempre di più nel corso degli anni. Da esperienze passate, so che questo è tremendamente offensivo per alcuni ma, ciononostante, non penso che dovrei scrivere una guida a come giocare oggi a \dnd{} senza affrontare il golfo temporale che separa noi da Gary Gygax.

\usubsection{Anche questo è \dnd{}}

La campagna tipica di \dnd{} è sempre saldamente ancorata alle tradizioni della narrativa d'avventura, un genere letterario che ha origine nel colonialismo, orientalismo, razzismo e autoritarismo del diciannovesimo secolo. Non aiuta il fatto che i primi autori e l'editore originale, \textbf{Tactical Study Rules} (più comunemente abbreviata in TSR), sorgessero assieme negli anni '70 da un background piuttosto conservatore ed estremamente Americano.

Il gioco costruito su queste premesse non è \textit{intenzionalmente} un veicolo di valori e attitudini coloniali, ma questi valori fondamentali erano condivisi dagli autori e formano la base di ciò che il gioco è. Osserva:

\paragraph{Gli avventurieri sono bianchi}: Non intendo che abbiano letteralmente la pelle bianca; gli avventurieri di \dnd{} sono parte della civiltà dominante e godono dei suoi vantaggi e del suo supporto. Le loro attività sono pienamente riconosciute dalla legge locale. Hanno l'iniziativa strategica e avanzano esplorando, generalmente essendo sull'offensiva e sapendo dove andare.

\paragraph{I mostri sono nativi}: Indipendentemente dalle storie delle specifiche campagne, rimane il fatto basilare che gli Umanoidi Malvagi Gygaxiani siano primariamente gruppi primitivi, che vivono letteralmente nelle caverne, molto lontani dalla civiltà avventuriera in termini di sofisticazione. Posseggono le caratteristiche che vengono universalmente attribuite dalla cultura dominante alle etnie marginalizzate: brutali, privi di emozioni, senza intelligenza, dominati dal Male.

\paragraph{L'avventura è conquista}: Gli avventurieri non bussano quando ti entrano in casa. Gli avventurieri ti uccideranno per i tuoi tesori e né le proprietà in uso né le tombe sono al sicuro. Considereranno un affronto se proverai a ostruire il loro passaggio o a difendere la tua proprietà. Non puoi ricorrere a un tribunale nella socità di origine dell'avventuriero; tutto quello che ti fa è legale.

Questa non è una caricatura, sto solo descrivendo quello che dice \dnd{} in termini molto tecnici e schietti: è un \textit{wargame} che simula la condotta pratica di una spedizione filibustiera tra comunità in uno stato di guerra costante. Potrei dire cose simili sulla natura delle guerre razziali, dell'esoticizzazione dello sconosciuto e molti altri temi, ma lo spirito è semplice: dal punto di vista delle tematiche, \dnd{} è identico alla letteratura d'avventura occidentale del ventesimo secolo. Raccoglie la sfida di essere "Tarzan: il gioco di ruolo" e comincia a correre.

È importante capire che la cultura storica del gioco non intende farne un qualche tipo di analogia diretta al colonialismo storico. Non c'è alcun cripto-KKK nascosto nel gioco (da quanto ne so, e ne so un po'). Ma è anche vero che il gioco non glorifica \textit{accidentalmente} avventure eroiche in un passato mitico in cui gli uomini erano eroi e c'erano mostri da uccidere. È un pezzo di tradizione culturale macho e patriarcale, nato dalla nozione che imparare a fare la guerra sia un passatempo inerentemente piacevole e studiare modi per cui gli eroici "noi" abbattono i mostruosi "loro" sia un obiettivo nobile.

Noi umani amiamo le storie di eroismo. Ma con la distanza culturale rispetto alle origini di \dnd che si allarga, la gloria che doveva essere sincera diventa sottile e satirica per molti di noi. Si rattoppa; ad esempio, spostando l'attenzione del gioco dalla filibusta dei dungeon alla guida della Grande Ribellione contro i Cattivi lo rende istantanteamente più appetibile per un pubblico contemporaneo, senza cambiare nulla della natura strutturale del gioco.

Questo è un argomento di cui si è parlato molto nell'hobby e ti invito a cercare diverse prospettive. Per la gran parte dell'era della TSR, \dnd{} corporate era normalmente pubblicato e descritto in un modo attentamente infantilizzato, al livello del Comic Code\translatornote{Abbreviazione di Comic Code Authority, un ente di (auto)censura del fumetto americano, stabilito negli anni '50 del secolo scorso dalle principali case editrici. Le rigide regole e la morale semplificata della CCA sono oggetto di parodia fin dalla sua nascita e ben dopo la sua scomparsa (2011). Il virgolettato che segue è una di queste.}: "è vietato che agenti delle forze dell'ordine muoiano come conseguenze delle attività del male" e via dicendo. Questo è\dots è la percezione di una certa visione del mondo di ciò che è importante, di ciò da cui bisogna guardarsi nello sviluppo del gioco. Ha senso se pensi che la letteratura d'avventura sia grandiosa fintantoché mettiamo bene in chiaro che gli Eroi uccidono solo i Cattivi.

Più avanti, è venuto di moda santificare il gioco con un riconoscimento di facciata dell'importanza dell'inclusività e delle politiche identitarie. Il gioco andrà bene se ci assicuriamo che gli avventurieri siano più internamente diversificati di quanto non fossero prima e se è genericamente piuttosto chiaro che "noi" include chiunque voglia unirsi e "loro" è molto, molto finto e mostruoso.

Ma siamo sempre a un "noi" che attacca un "loro" con varie giustificazioni che hanno senso per il pubblico attuale. È ancora un gioco dove la violenza risolve problemi e porta ricompense. È un \textit{wargame}, una fantasia eroica o, quantomeno, l'eco di una.

\usubsection{Cosa possiamo farci}

Non ho tutte le risposte su come il microcosmo si relazioni col macrocosmo, ma, partendo dalle mie convinzioni personali: penso contemporaneamente che sia importante capire noi stessi e la nostra storia, e che sia equamente importante arrivare alla conclusione finale che l'eroismo è una fantasia umana: è una storia che ci raccontiamo da soli, una delle tante storie che usiamo per dare senso a un mondo complesso e caotico.

Quindi la ragione per cui troviamo un significato in \dnd{} è che sì, è una fantasia sciovinistica, e questo piace agli umani: dipinge un'immagine di un mondo che necessita di eroi coraggiosi, un mondo assediato dall'oscurità, un mondo dove gloria e giustizia si possono ottenere in punta di spada. La brama esistenziale che rende l'intelligenza umana vulnerabile a questo richiamo è un argomento eccessivo per questa umile guida, ma il richiamo è reale. Io stesso sono assolutamente un fan della narrativa d'avventura, cresciuto a pane e letteratura fantasy, ma comunque: questa idea eroica è la fantasia. Non gli elfi e i maghi e la magia, l'idea di un "eroe" è singolarmente la parte più fantasiosa del costrutto letterario su cui riposa \dnd{}. Il mondo reale non funziona affatto così.

Ma non possiamo semplicemente concludere che le fantasie umane di avventura ed eroismo sono false coscienze e quindi vanno scartate; l'umanità non ce lo permette. Non puoi crescere un uomo affinché riconosca il male a meno di non far studiare quell'uomo e di chiedergli di osservare e capire. Coloro che ignorano la storia sono condannati a ripeterla, se proprio sono costretto a ripetere l'ovvio. Non puoi superare questo sogno eroico ignorandolo e sperando che se ne vada. Quella mia conclusione sull'eroismo che è un miraggio? Non puoi arrivarci se non conosci l'eroismo e non conosci te stesso. Tutto il resto, come vietare ai bambini di giocare alla guerra, è disarmo intellettuale.

Quindi, io suggerisco una relazione costruttivamente ironica col gioco: riconosci di cosa parla, capisci di cosa parla e impara da esso. Non sei il tuo personaggio, gli "avventurieri" nel gioco non hanno bisogno di essere moralmente giustificati ai tuoi occhi, Puoi giocare un \textit{conquistador}, prendere decisioni da \textit{conquistador} per ragioni da \textit{conquistador} senza essere un \textit{conquistador} tu stesso. Non serve che ti piaccia essere il tuo personaggio; ai nostri tavoli i giocatori sono spesso piuttosto aperti nel riconoscere le cose orribili che i loro personaggi talvolta fanno, guidati da avidità o necessità.

Così come non giochiamo a football per poter deridere la squadra perdente (elevando il bullismo al di sopra della sportività), non giochiamo a un gioco di filibusta fantastica per sognare ad occhi aperti l'utopia illecita del "diritto del più forte" e "l'unico orco buono è un orco morto". Al contrario: continuo a ripetere come la via del \textit{wargaming} sia una via di apprendimento, questo è uno degli argomenti che appariranno in verifica: comprendere la realtà socioeconomica che guida le strutture militari ed economiche che stanno alla base del tipo di industria di razziatori-rapinatori che \dnd racconta.

Giocare con bambini e amici è comune. Non credo che abbiamo un dovere di censurare l'argomento del gioco per il loro bene, ma le nostre risposte personali e le lezioni che ne traiamo sono una nostra responsabilità, quindi importa se glorifichi o metti in dubbio la spedizione filibustiera di \dnd{}.

Non penso davvero che gli umani abbiamo l'opzione di \textit{non} imparare da qualcosa complesso e appassionante come \dnd{}. La sola domanda è, che lezioni insegnerai?