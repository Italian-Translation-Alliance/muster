\usection{La Prima Dottrina del Dungeon}

A questo punto, è probabilmente piuttosto chiaro che il gioco si cura di sviluppare ed esercitare le abilità - la loro ridicola mancanza, dipende - dell'arte della guerra o, per usare un termine più moderno, della scienza militare. Potrebbe essere opportuno considerarne ora gli aspetti pratici.

\usubsection{Capire la Domanda}

L'errore più comune che vedo nell'esplorazione pratica dei dungeon sono i giocatori che non capiscono la domanda: è estremamente possibile giocare senza nemmeno realizzare che ci sia aspetta che tu pensi in maniera intelligente e prenda decisioni sagge. Per esempio, i giocatori potrebbero pensare che il loro compito sia semplicemente di farsi guidare dai suggerimento dell'arbitro e esperire il contenuto che gli viene mostrato, come se il gioco fosse un film. Potrebbero pensare che la cosa più importante da fare sia "interpretare" il loro personaggio come una personalità buffa o realistica. Ma se tutto quello di cui ti importa è far finta di aver paura dei goblin, non stai davvero giocando.

Combatti queste tendenze! Fidati dell'autenticità del gioco. Presumi che, quando ti trovi davanti alla porta di un dungeon, \textit{spetti davvero} a te decidere se aprire o meno quella porta. Questa è una scelta semplice e basilare, una che si potrebbe inquadrare come la scelta tra giocare e non giocare, e comunque ti consiglio: dovresti farla in maniera strategica. Quindi svegliati da questo stordimento ironico, prendi il controllo e metti davvero al lavoro il tuo cervello: è davvero nel miglior interesse del mio avventuriero fare questa scelta? E se no, perché no?

I giocatori che ingaggiano il gioco ricevono il cambio \textbf{dottrina}, le conclusioni cristallizzate della loro esperienza sul modo migliore di superare dungeon o altri tipi di avventure. Nel tempo, lo sviluppo della dottrina rende le operazioni più facili e più profonde, perché potrai concentrarti sempre di più su complessi problemi emergenti invece di quelli basilari.

\usubsection{Esempi di dottrina}

Alcuni esempi di lezioni basilari che mi piacciono, perché confido che possano applicarsi a numerose campagne, fintantoché si pratica l'esplorazione tradizionale dei dungeon. Questo è il tipo di apprendimento di buonsenso di cui consistono le parti atomiche della dottrina tattica:

\textbf{Controlla sempre il soffitto}

\textbf{Se puoi, procurati una lanterna}

\textbf{Esplora lentamente i dungeon sconosciuti}

\textbf{Scegli un leader e obbediscigli}

Quando stai costruendo la tua dottrina, presta attenzione all'idea di "buona pratica": questi tipo di idee su quello che dovresti o non dovresti fare nel dungeon sono importanti non perché ti garantiscano, in qualche modo, il successo, ma perché a parità di condizioni è probabile che saranno un'idea migliore dell'alternativa, come dimostrato dall'esperienza. Spesso la buona dottrina è direttamente un vantaggio tattico gratuito: le lezioni passate, guadagnate con fatica, che mostrano come l'opzione A sia strettamente superiore all'opzione B, quindi puoi direttamente scordarti l'opzione B. Come, ad esempio, controllare il soffitto; perché dovresti non volerlo fare, quando farlo ti evita di farti attaccare a sorpresa da un ragno gigante?

Normalmente, è tanto difficile stabilire una buona dottrina quanto lo è sapere quali idee sono buone al primo tentativo. Spesso vedo esploratori esperti che seguono il flusso, lasciando che i loro alleati commettano vecchi errori, o persino facendovisi intrappolare loro stessi. Potreste salvarvi a vicenda un po' di dolore parlando di dottrina e condividendo la vostra saggezza. Prendersi un po' di tempo per parlare delle lezioni apprese, magari all'inizio o alla fine delle sessioni, è un modo ridicolmente facile di migliorare il vostro gioco!

\begin{figure}[p!]
    \resizebox{\linewidth}{!}{
        \begin{tikzpicture}
            \node[text width=\linewidth, align=center] (T) at (-1, 0.25) {
                    \textsc{\LARGE{Dottrina e abilità}}\\
                    \textit{Questa è la parte effettivamente divertente del} wargaming\textit{, dove cerchi di imparare dall'esperienza e di formulare principi per contesti futuri.}
                };
                %Main flow
                \node[phase] (F1) at (-1,-1.5) {Il GM introduce lo scenario};
                \node[phase] (F2) at (-1,-4.6) {I giocatori risolvono lo scenario};
                \node[phase] (F3) at (-1,-8.6) {Formazione della dottrina};
                \node[action, text width=3cm] (A1) at (-1,-6.6) {Apprendimento};

                % D1
                \node (D1_1) at (-2, -3.6) {};
                \node (D1_2) at (-3.5, -3) {};
                \node (D1_3) at (-3, -4) {};
                \draw[->, ultra thick]
                    (D1_1) to[arc through={counterclockwise,(D1_2)}] (D1_3);
                \node[doctrine] (D1) at (-4.2,-3) {Abilità esterne};

                % D2
                \node[doctrine] (D2) at (-4.3,-6.6) {Applicazione della dottrina};

                \draw [->,line width=2pt] (F1) -- (F2);
                \draw [->,line width=2pt] (F2) -- (A1) -- (F3);
                \draw [-, line width=2pt] (F3.west) -- (-4.3, -8.6) -- (D2.south);
                \draw [->, line width=2pt] (D2.north) -- (-4.3, -4.6) -- (F2.west);

                \node[rectangle, align=left, text width=0.8\linewidth,font=\itshape] at (-2.4, -11) {Ci sono indubbiamente gloria e orgoglio nell'avere successo nel gioco, ma non bisognerebbe sottostimare la soddisfazione intellettuale del preparare una "guida" ben ponderata per le operazioni commando nei dungeon.};

        \end{tikzpicture}
    }
\end{figure}

\usubsection{Ruoli nel gruppo}

È una buona idea organizzare il gruppo di avventura con ruoli specializzati; aiuta i giocatori a concentrarsi, dà a ciascuno qualcosa da fare, è divertente e migliora le prestazioni in generale. Spesso i ruoli portano con sè dottrina con la loro mera esistenza.

Il modo migliore di assegnare dei ruoli è seguendo l'interesse dei giocatori: lascia che i giocatori attivi gravitino verso la gestione del gruppo. Uno dei principali punti di forza di \dnd come attività sociale, rispetto ad altri giochi di ruolo, è che va perfettamente bene che i singoli giocatori assumano ruoli piuttosto passivi, lasciando che gli altri facciano il grosso del gioco.

La seconda virtù dell'organizzazione del party è, ovviamente, far girare i ruoli. Ogni giocatore dovrebbe avere una chance di mettersi alla prova in cose diverse, nel tempo, è come cresciamo. Mentre fissare un calendario artificioso non funziona molto bene.

Nella mia esperienza, i ruoli importanti sono:

\paragraph{Leader del gruppo}, noto anche come portavoce, capitano, presidente, è il giocatore che gestisce la discussione del gruppo, raccoglie opinioni, formula decisioni su cui il gruppo vota e riferisce le manovre del gruppo all'arbitro. Un gruppo che fa male questo lavoro prende cattive decisioni, è confuso e gioca lentamente. Un gruppo con un buon presidente definisce in modo efficiente la direzione del gioco e fa succedere cose collettivamente, incontrando alla pari l'arbitro come fonte di direzione del gioco.

\paragraph{Mappatore}, spesso l'esploratore del gruppo, è il giocatore che fa domande all'arbitro sull'architettura del dungeon e mantiene l'immagine della situazione tattica del gruppo. Spesso il mappatore gestisce lo scambio concreto con l'arbitro riguardo al movimento e all'esplorazione del gruppo, facendo domande e assicurandosi di aver capito correttamente la descrizione del dungeon.

\paragraph{Quartiermastro} o sergente del gruppo o ufficiale logistico, è il giocatore che tiene traccia delle scorte del gruppo (torce, acqua, etc), del tempo a beneficio del gruppo, mantiene l'elenco del bottino e gestisce il contingente di PNG mercenari alleati. Risponde regolarmente a domande sullo stato del gioco che riguardano la composizione del gruppo, la situazione dell'illuminazione, l'ordine di marcia e via dicendo.

Ci sono altri ruoli che a volte si sviluppano e si possono sempre dividere o combinare ruoli in base alle persone che compongono il gruppo, ma penso che riempire questi tre slot sia la base ideale da cui partire. Non necessariamente in maniera dogmatica, insistendo su una distirbuzione formale dei ruoli, ma in un modo che garantisca che tutti i compiti vengano svolti in modo efficace.

\usubsection{Dottrina Basilare di Esplorazione}

Un piccolo gruppo all'interno di un dungeon sconosciuto dovrebbe, di default, far ricorso a un singolo esploratore che stia davanti al gruppo principale. L'esploratore sta almeno 10 piedi davanti all'effettiva linea del fronte e non si allontana oltre la portata di voce. Questo il paradigma dell'"esplorazione di prossimità", di contrasto con quella a distanza, una tecnica differente che potrste talvolta trovarvi a voler impiegare.

L'esploratore (e il gruppo che lo segue) dovrebbe procedere ad un passo da mappatura, abbastanza lentamente da poter notare pericoli e oppurtunità nelle grotte oscure e da prenderne nota. Dovrebbe usare la quantità minima di luce di cui ha bisogno, di volta in volta, per non attirare l'attenzione.

L'esploratore dovrebbe sincerarsi che l'arbitro sappia che l'esploratore sta sempre controllando tutte le direzioni mentre avanza, incluso il soffitto. Si fermerà e esaminerà attentamente elementi sospetti dell'architettura, come buchi nei muri. Prenderà pause per rimanere all'erta e permettere al gruppo di raggiungerlo a intervalli regolari.

Molte trappole nei dungeon possono essere evitate con l'uso dell'esplorazione. Alla fine, lo scopo dell'esplorazione è la gestione delle perdite: un pericolo potrebbe prendersi l'esploratore, ma il gruppo sopravviverà. Il gruppo potrebbe voler ruotare il ruolo di esploratore tra gli avventurieri per distribuire equamente il rischio.

\usubsection{Disciplina della Luce}

Lasciati a sè stessi, i giocatori trascureranno spesso le condizioni di luce del dungeon, rischiando di ricevere penalità dall'arbitro nelle situazioni in cui la luce è estremamente importante. Preallertare i mostri, perdersi indizi nascosti, penalità in combattimento e direttamente la perdita di illuminazione sono possibilità estremamente concrete.

Il gruppo dovrebbe addestrarsi a passare da condizioni di luce complete a scarsa luminosità e viceversa, spegnendo torce e coprendo lanterne quando necessario. Il gruppo dovrebbe puntare ad avere la miglior illuminazione possibile; lanterne al posto di torce e luce magica se possibile.

Numerosi membri del gruppo dovrebbero portare fonti di luce per ridurre ombre ingannevoli, migliorare in generale il livello di illuminazione e ridurre l'improvvisa confusione quando chi porta la luce si trova nel posto sbagliato al momento sbagliato. In un piccolo gruppo, portare una torcia a testa non è eccessivo.

Assicurati che, quando l'arbitro chiede lo stato delle luci durante un combattimento, state effettivamente mantenendo abbastanza illuminazione. Gettare tutti quanti le torce per estrarre le armi nel momento in cui inizia una battaglia, sprofondando la situazione in un'oscurità che avvantaggia solo i mostri, è un modo perfettamente legittimo di morire in un dungeon infernale.

Assicurati accompagnare con una fonte di luce personaggi che si occupano di cercare, studiare o effettuare qualsiasi altra osservazione accurata delle caratteristiche del dungeon. L'illuminazione incompleta è una ragione assolutamente non necessaria per non riuscire a comprendere dettagli importanti.

\usubsection{Dottrina di Ingresso nelle Stanze}

I gruppi muoiono perché non concordano e non si addestrano per l'azione organizzata. Assicurati in anticipo che i giocatori sappiano cosa ogni membro della squadra dovrebbe fare quando si spalanca una porta i un dungeon. Addestratevi in due metodi diversi:

\paragraph{L'entrata dura} si usa quando sospettate che ci siano ostili all'interno della stanza e volete prendere l'iniziativa e controllare la situazione in maniera rapida e decisiva. Il gruppo si organizza attorno all'entrata, la sfonda e carica verso l'interno della stanza seguendo un pattern di entrata pianificato. Questo è il modo migliore di entrare se, strategicamente, siete sull'offensiva e ci sono nemici nella stanza. Se c'è una trappola nella stanza, un'entrata dura non funzionerà.

\paragraph{L'entrata morbida} si usa quando non volete dare il via a ostilità o sospettate la presenza di trappole. Il gruppo si organizza per difendere la propria posizione nel corridoio al di fuori della porta (inizalmente a una distanza sufficiente a proteggersi da eventuali trappole), apre la porta silenzioamente, osserva e manda l'esploratore all'interno. Se scoppia una battaglia, si combatte difensivamente sulla strettoria della porta. L'entrata morbida andrebbe usata quando l'entrata dura non è indicata.

L'entrata peggiore è quella confusa, ovviamente; perde tanto contro una stanza piena di mostri che con una piena di trappole. Esplorare un dungeon prevede un sacco di situazioni in cui il gruppo ha l'iniziativa, con una porta verso l'ignoto che attende che la aprano. Almeno, abbiate la decenza di coordinarvi su quell'aspetto; aprirete centinaia di porte in \dnd, non avere una dottrina coerente per farlo sarebbe semplicemente triste.

\usubsection{Combattimento}

Un piccolo gruppo dovrebbe pianificare di combattere schermaglie difensive nel dungeon; non state cercando di vincere, state cercando di coprire la ritirata. Scegliete l'avanguardia e l'ordine di marcia di conseguenza.

Un gruppo più potente potrebbe individuare la propria formazione da combattimento per la guerra nel dungeon; qui non scenderò nei dettagli, ma la falange di lance regna da lungo tempo come forza dominante nelle strategie per i dungeon di basso livello. Arrivateci da soli.

È particolarmente utile essere coscienti del fatto che l'estetica "dungeon fantasy" che permea le visioni moderne di \dnd è in larga parte una narrazione stilizzata piuttosto che pratiche nate durante le partite; sono autori fantasy e artisti visivi che immaginano come dovrebbe apparire un pericoloso combattimento fantasy e non hanno alcuna particolare autorità quando dovrete capire come le cose funzionano per il vostro gruppo. Inoltre, questi modelli di comportamento sono, quasi senza eccezioni, quello che il nostro modello considererebbe di livello intermedio, che non si applicano per cercare di immaginare le tattiche per il gioco ai bassi livelli. Quindi, magari, non andare a cercare indizi su cosa funziona e come comportarti in film o libri fantasy famosi.

Sviluppate un'orgaizzazione del gruppo tenendo a mente la coordinazione durante il combattimento. Concordate segnali per attacco e ritirata, prendete in considerazione gli elementi di support (illuminazione, medici da campo, etc) e mettete a punto una catena di comando tattica.

I gruppi di avventurieri vengono spesso sconfitti perché gli manca la disciplina. È particolarmente pericoloso quando parte del gruppo rompe i ranghi per scappare, lasciando il resto a morire contro un nemico in forte vantaggio. Probabilmente la cosa più importante da fare è trovare il modo di coordinare la decisione di ritirarsi e imparare a farlo in un ordine che dia a tutti la possibilità di uscire vivi da una brutta situazione.

\usubsection{Infine, l'implicazione per l'arbitro}

Sì, le dottrine che consiglio sono buone idee al mio tavolo, con me come arbitro. E l'unica cosa che mantiene l'interoperabilità tra le nostre campagne è il processo di arbitraggio simulativo, noi due che guardiamo ai fatti sul campo - la situazione di gioco - e decidiamo cosa importa e cosa no. Questo magari ci porta a conclusioni simili su questioni più ovvie, ma anche a conclusioni diverse su altre.

A volte ti importa di un qualche dettaglio tattico che non mi interessa, a volte è il contrario. Questo non vuol dire che la dottrina sia priva di valore o "soggettiva", solo che il giocatore saggio deve fare attenzione nell'applicare dottrine apprese all'esterno in una nuova campagna. Fai esperimenti e vedi cosa funziona. Non sto cercando di distribuire trucchi dal funzionamento garantito, solo di illustrare lo scopo delle effettive domande che il gioco di esplorazione dei dungeon in \dnd offre. Giocare davvero al gioco significa considerare questi argomenti pragmatici.