\usection{Il Bandito Calabrone}

È ora di un'altra storia! Cercherò di catturare il feeling di giocare a \dnd come un gioco di avventura aperto e arbitrato. Nessuna trama pre-definita, solo sviluppi organici e sfide che portano a vittorie e fallimenti. In aggiunta, c'è una breve sbirciata a un tipo di scenario di \textit{wargaming} al di fuori del semplice dungeon.

Nell'estate del 2011 la nostra "campagna storica" era attiva da 10 sessioni. Eravamo già riusciti a far ammazzare un buon numero di avventurieri in \textsc{Tomb of the Iron God}, \textsc{Temple of the Ghoul}, \textsc{Tower of the Stargazer} e qualche missione secondaria, man mano che imparavamo il gioco. Alcuni personaggi avevano fortuna e abilità dalla loro ed erano riusciti a fare dei rispettabili progressi verso le soglie di PX per le rispettive classi e guardavano con gola il secondo livello.

Il gruppo era finito a viaggiare verso la più o meno remota cittadina di Pembrooktonshire per qualche semplice scusa, o forse semplicemente a caccia di nuovi spunti per avventure. Potrei persino aver suggerito ai giocatori che Io-il-GM avevo un po' di divertimento a portata di mano se fossero andati in quella specifica direzione sulla mappa (è passato un po' di tempo e la memoria mi tradisce). Quello che i giocatori non sapevano era che detta cittadina sarebbe stata il palco per l'antologia di avventure urbane \textsc{No Dignity in Death: Three Brides} di Jim Raggi\translatornote{In realtà l'autore è \textit{James} Raggi.}. Le avventure urbane sono una sfidante forma esperta di \dnd che varia parecchio rispetto alle più comuni avventure nei dungeon e questo sarebbe stato il primo contatto della campagna con esse.

La prima "sposa" di \textsc{No Dignity}, la \textsc{Small Town Murder}, è tanto diretta quando può esserlo un giallo, coinvolge un banchetto di nozze e l'improvvisa e violenta morte della sposa stessa. Come è tipico di questo tipo di avventure, richiede che i giocatori usino semplici abilità investigative per formare un gruppo di sospettati, interrogare i testimoni, individuare il criminale e denunciarlo. C'è un cast piuttosto ampio di PNG coinvolti nel grosso matrimonio, ma è fondamentalmente un semplice scenario di investigazione.

I giocatori non avevano alcuna dottrina tattica per anche solo iniziare a risolvere un omicidio misterioso, a parte qualsiasi esposizione avessero avuto a rappresentazioni letterarie, come Sherlock Holmes e simili, il che significava che avremmo probabilmente assistito a una certa quantità di fallimenti inefficienti e pochi risultati. Le parti più immediate dello scenario riguardavano principalmente ricevere spiegazioni e imparare alcune informazioni basilari sui PNG coinvolti: le famiglie di sposa e sposo e chiunque altro avesse incrociato la narrazione della situazione fatta dall'arbitro.

Al contrario che nell'esplorazione di dungeon, per cui la campagna stava sviluppando delle routine, gli avventuriari mostrarono scarsa inclinazione al tipo di appunti e investigazioni sistematiche che risolve gli scenari investigativi. Al tempo, 
nemmeno io, come arbitro, avevo molto il polso della cosa, ma mi ero rassegnato ad aspettarmi un estesa sequenza di fallimento. Se dovessi rifarlo oggi, introdurrei l'approccio da \textit{wargaming} alla risoluzione dei gialli in maniera molto più conscia invece di lanciarli nella parte profonda della piscina (anche se si trattava di uno scenario piuttosto gentile con gli inesperti).

Parte del terreno di manovra di \textsc{Small Town Murder} è che gli avventurieri hanno un investigatore rivale nella forma del Cavaliere della Scienza, Tiberius Tucca. Detto gentiluomo arriva sulla scena del crimine con spavalderia e pomposità, pronto ad accusare gli zingari palesemente incastrati per il sanguinoso crimine. Non c'è bisogno di investigare quando sei lo sfacciato antagonista antagonista prefatto.

Quindi i giocatori hanno il campo, che fare? Sipi ha un'idea originale e proattiva; piuttosto letteraria e quindi non necessariamente fattibile, ma assolutamente meglio che non fare niente. Il gruppo avrebbe evitato di investigare quello che sembrava un macello tremendamente convoluto di motivazioni ed alibi, dedicandosi invece a dare direttamente la caccia all'assassino!

Il personaggio di Sipi, un Guerriero di nome Hans Krüger, era una specie di capo del gruppo in questo contesto, quindi non ha fatto particolare fatica a convincere gli altri avventurieri a seguire il suo esempio: siccome il Cavaliere della Scienza era \textit{ovviamente} l'assassino (perché era il personaggio meno gradevole introdotto fin'ora, suppongo?), il gruppo sarebbe dovuto arrivargli addosso e fingere veramente tanto di aver già risolto il crimine ed essere pronto a fare rapporto allo sceriffo locale il mattino seguente. Credo ci fossero anche alcuni vaghi riferimenti all'aver trovato un diario o qualcosa del genere.

Un lettore familiare con il genere letterario del giallo sa a cosa sarebbe dovuto servire: l'omicida, così provocato, avrebbe tentato nottetempo di zittire gli avventurieri in un disperato tentativo di evitare di essere denunciato allo sceriffo. Sospetto che questo sia un'idea di trama talmente comune da avere un nome da qualche parte. Una specie di stereotipo nella narrativa gialla. Ma avrebbe funzionato in un contesto di \textit{wargaming}?

I giocatori avevano puntato la loro indagine sul fatto che il Cavaliere della Scienza fosse l'assassino, con la forza del loro istinto (o della lettura dell'arbitro, come preferite). Alla fine, gli avventurieri si sono rivelati terribilmente fortunati in questa istanza: l'assassinio era stato commesso da Faustius Germanicus, lo scudiero del cavaliere. Un personaggio a malapena menzionato come qualcuno che esisteva. E, cosa importantissima, qualcuno che era proprio lì, ad ascoltare il tentativo di manipolazione sociale di Hans mentre tentava di attirare il cavaliere in trappola.

Gli eventi avvenuti quella notte nella daverna dove risiedevano gli avventurieri furono piuttosto ridicoli, quando il Bandito Calabrone, un personaggio letterario noto solo per romanzi d'avventura di bassa lega ad avventurieri che superavano i loro tiri di INT, apparve per visitare Hans e amici, che si aspettavano un tentativo di assassinio notturno. Perché il Bandito Calabrone? Beh, era esattamente il \textit{punto} di tutta questa avventura misteriosa così attentamente preparata. Il padre della sposa ha una stamperia che sta lavorando al prossimo episodio della pessima letteratura sul Bandito Calabrone, lo scudiero del cavaliere è un grande fan, passa tutto il giorno a leggere il manoscritto, si innamora della figlia del tipografo, ha manie di grandezza\dots

I giocatori si erano, ovviamente, persi tutto questo, perchè a chi serve veramente interrogare i testimoni o, diciamo, investigare qualsiasi cosa. Arriva sul posto, sostieni rumorosamente di aver risolto il crimine e ritirati in attesa che l'assassino si riveli.

La cosa più folle è che Hans riuscì, intuitivamente (fondamentalmente affidandosi alle capacità sociali e culturali del giocatore), ad inquadrare la sua narrazione di una "rivelazione imminente" in una luce credibile nonostante fosse rivolta alla persona sbagliata. Bello spettacolo, applausi.

Quindi gli avventurieri confrontarono il Bandito Calabrone, letteralmente chiudendogli la finestra alle spalle dopo che era entrato. Ancora convinti che fosse il Cavaliere della Scienza finché non riuscirono a smascherare l'uomo nonostante la sua feroce resistenza, scoprirono che non era altri che Faustius Germanicus lo scudiero (chi il cosa? Beh, imparate a investigare\dots)

Come spesso succede nella caotica simulazione del combattimento di \dnd, Faustius riuscì a sfuggire rocambolescamente dalla taverna, nonostante gli avventurieri si fossero preparati al suo arrivo e tentò di fuggire a cavallo nella notte. I giocatori furono entusiasti quando Hans incoronò la sua interpretazione di un eroe giallo inseguendo il Bandito come un professionista. Ero piuttosto sicuro che lo scudiero fosse scappato a fuggire quando era riuscito, con l'inganno, a lasciare la locanda, ma no; l'ostinazione e i dadi decisero altrimenti.

La buffa conclusione dell'intera storia arrivò quando gli avventurieri decisero di consegnare il loro prigioniero al Cavaliere della Scienza, una specie di poliziotto federale se confrontato con il più locale sceriffo disponibile a Pembrooktonshire. Faustius doveva essere stato piuttosto vicino al buffone, visto che i dadi indicano che riuscì a sfuggire alla sua custodia. Forse il cavaliere voleva risparmiarsi l'imbarazzo di un processo contro il suo stesso scudiero. Be, non importa agli eroi del giorno: gli avventurieri furono ricompensati piuttosto bene dalla famiglia zingara che avevano salvato dalla giustizia dei bianchi scoprendo il vero, e ora fuorilegge, assassino.

Tutto questo è avvenuto in una singola sessione di gioco, rivelandosi un'avventura piuttosto compatta in termini di tempo. Un po' come dire "e se provassimo a stanarli col fumo?" nell'esplorazione di un dungeon, ma in un tipo di scenario investigativo.

In una campagna estesa e su larga scala di \dnd nulla termina davvero, ovviamente, quindi la storia ha il suo epilogo farsesco. Vediamo:

Hans Krüger era la stella nascente della campagna, dopo aver guidato il suo gruppo verso la vittoria in questa avventura. Arrivò a partecipare ai \textsc{The Great Games} di Pembrooktonshire, la "seconda sposa" di \textsc{No Dignity}; prendendo il posto di uno dei partecipanti sulla forza della nuova reputazione del gruppo. Vincendo nuova gloria, Hans si portò a casa un matrimonio con la figlia di un ricco produttore di tapperi locale. Questo lo portà a viaggiare ad Amsterdam a caccia di alcuni segreti persiani sulla produzione dei tappeti, nuove avventure e infine fondare la propria compagnia mercenaria finanziata dal suocero.

Nella sessione \#84 della campagna, oltre 70 sessioni da quando Faustius Germanicus fuggì nella notte, tornò di nuovo a infestare le nostre avventure! Hans, ora un Guerriero/Duellante di sesto livello, era completamente impelagato nelle Guerre Italiane per conto dell'imperatore, quando il suo entourage incappò in Faustius Germanicus alla guida di un gruppo di banditi nelle campagne di Ferrera\translatornote{Suppongo fosse Ferrara}. Faustius, a questo punto un Guerriero di secondo livello, non era entusiasta di venir riportato alla realtà da Hans per la seconda volta. Questo secondo incontro si rivelò altrettanto inconcludente, penso, con nessuna fazione davvero intenzionata a spingere più di tanto sull'argomento. Da quel che ne so, Faustius sta ancora inseguendo il suo sogno di diventare un glorioso ed eroico bandito seguendo le orme del Bandito Calabrone.